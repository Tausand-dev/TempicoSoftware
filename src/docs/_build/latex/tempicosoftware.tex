%% Generated by Sphinx.
\def\sphinxdocclass{report}
\documentclass[letterpaper,10pt,english]{sphinxmanual}
\ifdefined\pdfpxdimen
   \let\sphinxpxdimen\pdfpxdimen\else\newdimen\sphinxpxdimen
\fi \sphinxpxdimen=.75bp\relax
\ifdefined\pdfimageresolution
    \pdfimageresolution= \numexpr \dimexpr1in\relax/\sphinxpxdimen\relax
\fi
%% let collapsible pdf bookmarks panel have high depth per default
\PassOptionsToPackage{bookmarksdepth=5}{hyperref}

\PassOptionsToPackage{booktabs}{sphinx}
\PassOptionsToPackage{colorrows}{sphinx}

\PassOptionsToPackage{warn}{textcomp}
\usepackage[utf8]{inputenc}
\ifdefined\DeclareUnicodeCharacter
% support both utf8 and utf8x syntaxes
  \ifdefined\DeclareUnicodeCharacterAsOptional
    \def\sphinxDUC#1{\DeclareUnicodeCharacter{"#1}}
  \else
    \let\sphinxDUC\DeclareUnicodeCharacter
  \fi
  \sphinxDUC{00A0}{\nobreakspace}
  \sphinxDUC{2500}{\sphinxunichar{2500}}
  \sphinxDUC{2502}{\sphinxunichar{2502}}
  \sphinxDUC{2514}{\sphinxunichar{2514}}
  \sphinxDUC{251C}{\sphinxunichar{251C}}
  \sphinxDUC{2572}{\textbackslash}
\fi
\usepackage{cmap}
\usepackage[T1]{fontenc}
\usepackage{amsmath,amssymb,amstext}
\usepackage{babel}



\usepackage{tgtermes}
\usepackage{tgheros}
\renewcommand{\ttdefault}{txtt}



\usepackage[Bjarne]{fncychap}
\usepackage{sphinx}

\fvset{fontsize=auto}
\usepackage{geometry}


% Include hyperref last.
\usepackage{hyperref}
% Fix anchor placement for figures with captions.
\usepackage{hypcap}% it must be loaded after hyperref.
% Set up styles of URL: it should be placed after hyperref.
\urlstyle{same}

\addto\captionsenglish{\renewcommand{\contentsname}{Contents:}}

\usepackage{sphinxmessages}
\setcounter{tocdepth}{0}


    \usepackage{hyperref}
    \hypersetup{
        colorlinks=true,
        linkcolor=blue,
        filecolor=magenta,      
        urlcolor=blue,
        pdftitle={Título del Proyecto},
        pdfauthor={Tu Nombre},
        pdfkeywords={palabras clave, del, proyecto},
    }
    

\title{Tempico Software}
\date{Jan 12, 2025}
\release{1.1.0}
\author{David Guzman, Joan Amaya}
\newcommand{\sphinxlogo}{\vbox{}}
\renewcommand{\releasename}{Release}
\makeindex
\begin{document}

\ifdefined\shorthandoff
  \ifnum\catcode`\=\string=\active\shorthandoff{=}\fi
  \ifnum\catcode`\"=\active\shorthandoff{"}\fi
\fi

\pagestyle{empty}
\sphinxmaketitle
\pagestyle{plain}
\sphinxtableofcontents
\pagestyle{normal}
\phantomsection\label{\detokenize{index::doc}}
\begin{figure}[htbp]
\centering

\noindent\sphinxincludegraphics[scale=0.6]{{image}.png}
\end{figure}



\sphinxAtStartPar
Tempico Software is a program that allows the use of Tausand Tempico TP1004 devices to simplify the creation of histograms and FLIM measurements. For more information, visit tausand.com.

\sphinxstepscope


\chapter{FLIMGraphics module}
\label{\detokenize{FLIMGraphics:module-FLIMGraphics}}\label{\detokenize{FLIMGraphics:flimgraphics-module}}\label{\detokenize{FLIMGraphics::doc}}\index{module@\spxentry{module}!FLIMGraphics@\spxentry{FLIMGraphics}}\index{FLIMGraphics@\spxentry{FLIMGraphics}!module@\spxentry{module}}\index{FLIMGraphic (class in FLIMGraphics)@\spxentry{FLIMGraphic}\spxextra{class in FLIMGraphics}}

\begin{fulllineitems}
\phantomsection\label{\detokenize{FLIMGraphics:FLIMGraphics.FLIMGraphic}}
\pysigstartsignatures
\pysiglinewithargsret{\sphinxbfcode{\sphinxupquote{class\DUrole{w,w}{  }}}\sphinxcode{\sphinxupquote{FLIMGraphics.}}\sphinxbfcode{\sphinxupquote{FLIMGraphic}}}{\sphinxparam{\DUrole{n,n}{comboBoxStartChannel}\DUrole{p,p}{:}\DUrole{w,w}{  }\DUrole{n,n}{QComboBox}}\sphinxparamcomma \sphinxparam{\DUrole{n,n}{comboBoxStopChannel}\DUrole{p,p}{:}\DUrole{w,w}{  }\DUrole{n,n}{QComboBox}}\sphinxparamcomma \sphinxparam{\DUrole{n,n}{graphicFrame}\DUrole{p,p}{:}\DUrole{w,w}{  }\DUrole{n,n}{QFrame}}\sphinxparamcomma \sphinxparam{\DUrole{n,n}{startButton}\DUrole{p,p}{:}\DUrole{w,w}{  }\DUrole{n,n}{QPushButton}}\sphinxparamcomma \sphinxparam{\DUrole{n,n}{stopButton}\DUrole{p,p}{:}\DUrole{w,w}{  }\DUrole{n,n}{QPushButton}}\sphinxparamcomma \sphinxparam{\DUrole{n,n}{initialParametersButton}\DUrole{p,p}{:}\DUrole{w,w}{  }\DUrole{n,n}{QPushButton}}\sphinxparamcomma \sphinxparam{\DUrole{n,n}{clearButton}\DUrole{p,p}{:}\DUrole{w,w}{  }\DUrole{n,n}{QPushButton}}\sphinxparamcomma \sphinxparam{\DUrole{n,n}{saveDataButton}\DUrole{p,p}{:}\DUrole{w,w}{  }\DUrole{n,n}{QPushButton}}\sphinxparamcomma \sphinxparam{\DUrole{n,n}{savePlotButton}\DUrole{p,p}{:}\DUrole{w,w}{  }\DUrole{n,n}{QPushButton}}\sphinxparamcomma \sphinxparam{\DUrole{n,n}{statusLabel}\DUrole{p,p}{:}\DUrole{w,w}{  }\DUrole{n,n}{QLabel}}\sphinxparamcomma \sphinxparam{\DUrole{n,n}{pointLabel}\DUrole{p,p}{:}\DUrole{w,w}{  }\DUrole{n,n}{QLabel}}\sphinxparamcomma \sphinxparam{\DUrole{n,n}{binWidthComboBox}\DUrole{p,p}{:}\DUrole{w,w}{  }\DUrole{n,n}{QComboBox}}\sphinxparamcomma \sphinxparam{\DUrole{n,n}{numberBins}\DUrole{p,p}{:}\DUrole{w,w}{  }\DUrole{n,n}{QComboBox}}\sphinxparamcomma \sphinxparam{\DUrole{n,n}{functionComboBox}\DUrole{p,p}{:}\DUrole{w,w}{  }\DUrole{n,n}{QComboBox}}\sphinxparamcomma \sphinxparam{\DUrole{n,n}{numberMeasurementsSpinBox}\DUrole{p,p}{:}\DUrole{w,w}{  }\DUrole{n,n}{QSpinBox}}\sphinxparamcomma \sphinxparam{\DUrole{n,n}{totalMeasurements}\DUrole{p,p}{:}\DUrole{w,w}{  }\DUrole{n,n}{QLabel}}\sphinxparamcomma \sphinxparam{\DUrole{n,n}{totalStart}\DUrole{p,p}{:}\DUrole{w,w}{  }\DUrole{n,n}{QLabel}}\sphinxparamcomma \sphinxparam{\DUrole{n,n}{totalTime}\DUrole{p,p}{:}\DUrole{w,w}{  }\DUrole{n,n}{QLabel}}\sphinxparamcomma \sphinxparam{\DUrole{n,n}{timeRange}\DUrole{p,p}{:}\DUrole{w,w}{  }\DUrole{n,n}{QLabel}}\sphinxparamcomma \sphinxparam{\DUrole{n,n}{device}}\sphinxparamcomma \sphinxparam{\DUrole{n,n}{applyButton}\DUrole{p,p}{:}\DUrole{w,w}{  }\DUrole{n,n}{QPushButton}}\sphinxparamcomma \sphinxparam{\DUrole{n,n}{parameterTable}\DUrole{p,p}{:}\DUrole{w,w}{  }\DUrole{n,n}{QTableWidget}}\sphinxparamcomma \sphinxparam{\DUrole{n,n}{MainWindow}}\sphinxparamcomma \sphinxparam{\DUrole{n,n}{timerStatus}\DUrole{p,p}{:}\DUrole{w,w}{  }\DUrole{n,n}{QTimer}}}{}
\pysigstopsignatures
\sphinxAtStartPar
Bases: \sphinxcode{\sphinxupquote{object}}

\sphinxAtStartPar
Class responsible for the logic and functionality of the FLIM (Fluorescence Lifetime Measurement) window.

\sphinxAtStartPar
This class manages the creation and updating of graphical plots related to FLIM, controls user interactions with various 
buttons, and ensures that measurements are taken from the specified channels. The class handles:
\sphinxhyphen{} Initialization of the graphical interface elements.
\sphinxhyphen{} Starting and stopping of measurements.
\sphinxhyphen{} Saving of data and plots.
\sphinxhyphen{} Displaying parameters of the measurements, such as the number of measurements taken, and the percentage of completion.
\begin{quote}\begin{description}
\sphinxlineitem{Parameters}\begin{itemize}
\item {} 
\sphinxAtStartPar
\sphinxstyleliteralstrong{\sphinxupquote{comboBoxStartChannel}} \textendash{} The combo box for selecting the start channel (QComboBox).

\item {} 
\sphinxAtStartPar
\sphinxstyleliteralstrong{\sphinxupquote{comboBoxStopChannel}} \textendash{} The combo box for selecting the stop channel (QComboBox).

\item {} 
\sphinxAtStartPar
\sphinxstyleliteralstrong{\sphinxupquote{graphicFrame}} \textendash{} The frame that holds the graphical plot (QFrame).

\item {} 
\sphinxAtStartPar
\sphinxstyleliteralstrong{\sphinxupquote{startButton}} \textendash{} The button to start the measurements (QPushButton).

\item {} 
\sphinxAtStartPar
\sphinxstyleliteralstrong{\sphinxupquote{stopButton}} \textendash{} The button to stop the measurements (QPushButton).

\item {} 
\sphinxAtStartPar
\sphinxstyleliteralstrong{\sphinxupquote{initialParametersButton}} \textendash{} The button to configure initial parameters (QPushButton).

\item {} 
\sphinxAtStartPar
\sphinxstyleliteralstrong{\sphinxupquote{clearButton}} \textendash{} The button to clear the graphs and data (QPushButton).

\item {} 
\sphinxAtStartPar
\sphinxstyleliteralstrong{\sphinxupquote{saveDataButton}} \textendash{} The button to save the measurement data (QPushButton).

\item {} 
\sphinxAtStartPar
\sphinxstyleliteralstrong{\sphinxupquote{savePlotButton}} \textendash{} The button to save the graph plot (QPushButton).

\item {} 
\sphinxAtStartPar
\sphinxstyleliteralstrong{\sphinxupquote{statusLabel}} \textendash{} The label showing the current status (QLabel).

\item {} 
\sphinxAtStartPar
\sphinxstyleliteralstrong{\sphinxupquote{pointLabel}} \textendash{} The label displaying the current point in the measurement (QLabel).

\item {} 
\sphinxAtStartPar
\sphinxstyleliteralstrong{\sphinxupquote{binWidthComboBox}} \textendash{} The combo box for selecting the bin width (QComboBox).

\item {} 
\sphinxAtStartPar
\sphinxstyleliteralstrong{\sphinxupquote{numberBins}} \textendash{} The combo box for selecting the number of bins (QComboBox).

\item {} 
\sphinxAtStartPar
\sphinxstyleliteralstrong{\sphinxupquote{functionComboBox}} \textendash{} The combo box for selecting a function (QComboBox).

\item {} 
\sphinxAtStartPar
\sphinxstyleliteralstrong{\sphinxupquote{numberMeasurementsSpinBox}} \textendash{} The spin box for specifying the number of measurements (QSpinBox).

\item {} 
\sphinxAtStartPar
\sphinxstyleliteralstrong{\sphinxupquote{totalMeasurements}} \textendash{} The label showing the total number of measurements (QLabel).

\item {} 
\sphinxAtStartPar
\sphinxstyleliteralstrong{\sphinxupquote{totalStart}} \textendash{} The label showing the number of measurements taken from the start channel (QLabel).

\item {} 
\sphinxAtStartPar
\sphinxstyleliteralstrong{\sphinxupquote{totalTime}} \textendash{} The label showing the total time of measurements (QLabel).

\item {} 
\sphinxAtStartPar
\sphinxstyleliteralstrong{\sphinxupquote{timeRange}} \textendash{} The label showing the time range for the measurements (QLabel).

\item {} 
\sphinxAtStartPar
\sphinxstyleliteralstrong{\sphinxupquote{device}} \textendash{} The connected Tempico device used for measurements.

\item {} 
\sphinxAtStartPar
\sphinxstyleliteralstrong{\sphinxupquote{applyButton}} \textendash{} The button for applying selected parameters (QPushButton).

\item {} 
\sphinxAtStartPar
\sphinxstyleliteralstrong{\sphinxupquote{parameterTable}} \textendash{} The table displaying the parameters of the measurements (QTableWidget).

\item {} 
\sphinxAtStartPar
\sphinxstyleliteralstrong{\sphinxupquote{MainWindow}} \textendash{} The main window that holds the application UI elements.

\item {} 
\sphinxAtStartPar
\sphinxstyleliteralstrong{\sphinxupquote{timerStatus}} \textendash{} The timer used for periodic updates (QTimer).

\end{itemize}

\sphinxlineitem{Returns}
\sphinxAtStartPar
None

\end{description}\end{quote}
\index{applyAction() (FLIMGraphics.FLIMGraphic method)@\spxentry{applyAction()}\spxextra{FLIMGraphics.FLIMGraphic method}}

\begin{fulllineitems}
\phantomsection\label{\detokenize{FLIMGraphics:FLIMGraphics.FLIMGraphic.applyAction}}
\pysigstartsignatures
\pysiglinewithargsret{\sphinxbfcode{\sphinxupquote{applyAction}}}{}{}
\pysigstopsignatures
\sphinxAtStartPar
Captures the action of the ‘apply’ button, calculates the selected fit, and stores the resulting parameters.

\sphinxAtStartPar
This function determines which fitting function is selected from \sphinxtitleref{functionComboBox} and attempts to calculate
the fit using the measured time and data. If the parameters can be determined, they are stored in \sphinxtitleref{FitParameters}.
If not, a dialog informs the user that the parameters could not be determined.

\sphinxAtStartPar
Variables assignment:
\sphinxhyphen{} The parameters \sphinxtitleref{i0}, \sphinxtitleref{tau0}, and other fit\sphinxhyphen{}specific variables are calculated using the corresponding fit function.
\sphinxhyphen{} If the fit is successful, the rounded parameters are stored in \sphinxtitleref{FitParameters}.
\sphinxhyphen{} If the fit fails or returns “Undefined”, a message box is shown to inform the user of the error.
\begin{quote}\begin{description}
\sphinxlineitem{Returns}
\sphinxAtStartPar
None

\end{description}\end{quote}

\end{fulllineitems}

\index{applyInitialDialog() (FLIMGraphics.FLIMGraphic method)@\spxentry{applyInitialDialog()}\spxextra{FLIMGraphics.FLIMGraphic method}}

\begin{fulllineitems}
\phantomsection\label{\detokenize{FLIMGraphics:FLIMGraphics.FLIMGraphic.applyInitialDialog}}
\pysigstartsignatures
\pysiglinewithargsret{\sphinxbfcode{\sphinxupquote{applyInitialDialog}}}{}{}
\pysigstopsignatures
\sphinxAtStartPar
Updates the initial parameters based on user input from the parameter dialog.

\sphinxAtStartPar
This function is called when the user clicks the “Apply” button in the parameter dialog. 
It retrieves the values from the input fields corresponding to the selected fitting function 
and assigns them to the appropriate instance variables.

\sphinxAtStartPar
The following parameters are updated based on the selected fit type:
\sphinxhyphen{} For Exponential fit:
\sphinxhyphen{} I0 (initial value)
\sphinxhyphen{} tau0 (decay constant)
\sphinxhyphen{} For Kohlrausch fit:
\sphinxhyphen{} I0 (initial value)
\sphinxhyphen{} tau0 (decay constant)
\sphinxhyphen{} beta (exponent)
\sphinxhyphen{} For Shifted Exponential fit:
\sphinxhyphen{} I0 (initial value)
\sphinxhyphen{} tau0 (decay constant)
\sphinxhyphen{} alpha (shift parameter)
\sphinxhyphen{} b (baseline value)
\sphinxhyphen{} For Double Exponential fit:
\sphinxhyphen{} I0 (initial value)
\sphinxhyphen{} tau0 (first decay constant)
\sphinxhyphen{} tau1 (second decay constant)
\sphinxhyphen{} alpha (mixing coefficient)

\sphinxAtStartPar
Variables assignment:
\sphinxhyphen{} The updated values are assigned to their respective initial parameter variables.
\sphinxhyphen{} A boolean flag indicating that initial parameters have changed is set to True for the relevant fit type.
\begin{quote}\begin{description}
\sphinxlineitem{Returns}
\sphinxAtStartPar
None

\end{description}\end{quote}

\end{fulllineitems}

\index{calculateR2() (FLIMGraphics.FLIMGraphic method)@\spxentry{calculateR2()}\spxextra{FLIMGraphics.FLIMGraphic method}}

\begin{fulllineitems}
\phantomsection\label{\detokenize{FLIMGraphics:FLIMGraphics.FLIMGraphic.calculateR2}}
\pysigstartsignatures
\pysiglinewithargsret{\sphinxbfcode{\sphinxupquote{calculateR2}}}{\sphinxparam{\DUrole{n,n}{data}}\sphinxparamcomma \sphinxparam{\DUrole{n,n}{fitData}}}{}
\pysigstopsignatures
\sphinxAtStartPar
Calculates the R\(\sp{\text{2}}\) value based on the observed and fitted data.
\begin{quote}\begin{description}
\sphinxlineitem{Parameters}\begin{itemize}
\item {} 
\sphinxAtStartPar
\sphinxstyleliteralstrong{\sphinxupquote{data}} \textendash{} The observed data points (array\sphinxhyphen{}like).

\item {} 
\sphinxAtStartPar
\sphinxstyleliteralstrong{\sphinxupquote{fitData}} \textendash{} The fitted data points (array\sphinxhyphen{}like).

\end{itemize}

\sphinxlineitem{Returns}
\sphinxAtStartPar
The R\(\sp{\text{2}}\) value indicating the goodness of fit (float).

\end{description}\end{quote}

\end{fulllineitems}

\index{changeCreatedStatus() (FLIMGraphics.FLIMGraphic method)@\spxentry{changeCreatedStatus()}\spxextra{FLIMGraphics.FLIMGraphic method}}

\begin{fulllineitems}
\phantomsection\label{\detokenize{FLIMGraphics:FLIMGraphics.FLIMGraphic.changeCreatedStatus}}
\pysigstartsignatures
\pysiglinewithargsret{\sphinxbfcode{\sphinxupquote{changeCreatedStatus}}}{}{}
\pysigstopsignatures
\sphinxAtStartPar
Notifies that the measurement thread has been created.

\sphinxAtStartPar
This function sets the status of thread creation to True.
\begin{quote}\begin{description}
\sphinxlineitem{Returns}
\sphinxAtStartPar
None

\end{description}\end{quote}

\end{fulllineitems}

\index{changeFunction() (FLIMGraphics.FLIMGraphic method)@\spxentry{changeFunction()}\spxextra{FLIMGraphics.FLIMGraphic method}}

\begin{fulllineitems}
\phantomsection\label{\detokenize{FLIMGraphics:FLIMGraphics.FLIMGraphic.changeFunction}}
\pysigstartsignatures
\pysiglinewithargsret{\sphinxbfcode{\sphinxupquote{changeFunction}}}{}{}
\pysigstopsignatures
\sphinxAtStartPar
Updates the parameter values in the UI based on the selected function in the functionComboBox.

\sphinxAtStartPar
This function checks the current fitting function (e.g., “ExpDecay”, “Kohlrausch”, “ShiftedExponential”, or 
“DoubleExponential”) and updates the parameter labels accordingly. If the parameters have already been calculated,
it will display the values along with their uncertainties and units. If the parameters are not available, it sets 
the values to “Undefined”.

\sphinxAtStartPar
Variables assignment:
\sphinxhyphen{} Parameters like “I0”, “τ0”, “β”, “α”, “b”, “R\textasciicircum{}2”, and their respective uncertainties are assigned using 
\sphinxtitleref{insertParameters()} based on the fitting model.
\sphinxhyphen{} Units are assigned through the \sphinxtitleref{self.units} attribute for certain parameters (e.g., τ0).

\sphinxAtStartPar
Behavior by function:
\sphinxhyphen{} ExpDecay: Updates \sphinxtitleref{I0}, \sphinxtitleref{τ0}, and \sphinxtitleref{R\textasciicircum{}2}.
\sphinxhyphen{} Kohlrausch: Updates \sphinxtitleref{I0}, \sphinxtitleref{τ0}, \sphinxtitleref{β}, and \sphinxtitleref{R\textasciicircum{}2}.
\sphinxhyphen{} ShiftedExponential: Updates \sphinxtitleref{I0}, \sphinxtitleref{τ0}, \sphinxtitleref{α}, \sphinxtitleref{b}, and \sphinxtitleref{R\textasciicircum{}2}.
\sphinxhyphen{} DoubleExponential: Updates \sphinxtitleref{I0}, \sphinxtitleref{τ0}, \sphinxtitleref{τ1}, \sphinxtitleref{α}, and \sphinxtitleref{R\textasciicircum{}2}.
\begin{quote}\begin{description}
\sphinxlineitem{Returns}
\sphinxAtStartPar
None

\end{description}\end{quote}

\end{fulllineitems}

\index{changeStatusColor() (FLIMGraphics.FLIMGraphic method)@\spxentry{changeStatusColor()}\spxextra{FLIMGraphics.FLIMGraphic method}}

\begin{fulllineitems}
\phantomsection\label{\detokenize{FLIMGraphics:FLIMGraphics.FLIMGraphic.changeStatusColor}}
\pysigstartsignatures
\pysiglinewithargsret{\sphinxbfcode{\sphinxupquote{changeStatusColor}}}{\sphinxparam{\DUrole{n,n}{color}}}{}
\pysigstopsignatures
\sphinxAtStartPar
Changes the color of the point in the status bar. Each color is assigned a specific numeric value.
\begin{quote}\begin{description}
\sphinxlineitem{Parameters}
\sphinxAtStartPar
\sphinxstyleliteralstrong{\sphinxupquote{color}} \textendash{} The numeric value corresponding to the desired color (int).
\sphinxhyphen{} 0: Gray
\sphinxhyphen{} 1: Green
\sphinxhyphen{} 2: Yellow
\sphinxhyphen{} 3: Orange

\sphinxlineitem{Returns}
\sphinxAtStartPar
None

\end{description}\end{quote}

\end{fulllineitems}

\index{changeStatusLabel() (FLIMGraphics.FLIMGraphic method)@\spxentry{changeStatusLabel()}\spxextra{FLIMGraphics.FLIMGraphic method}}

\begin{fulllineitems}
\phantomsection\label{\detokenize{FLIMGraphics:FLIMGraphics.FLIMGraphic.changeStatusLabel}}
\pysigstartsignatures
\pysiglinewithargsret{\sphinxbfcode{\sphinxupquote{changeStatusLabel}}}{\sphinxparam{\DUrole{n,n}{textValue}}}{}
\pysigstopsignatures
\sphinxAtStartPar
Changes the text of the status label.
\begin{quote}\begin{description}
\sphinxlineitem{Parameters}
\sphinxAtStartPar
\sphinxstyleliteralstrong{\sphinxupquote{textValue}} \textendash{} The text to set for the status label (str).

\sphinxlineitem{Returns}
\sphinxAtStartPar
None

\end{description}\end{quote}

\end{fulllineitems}

\index{checkDeviceStatus() (FLIMGraphics.FLIMGraphic method)@\spxentry{checkDeviceStatus()}\spxextra{FLIMGraphics.FLIMGraphic method}}

\begin{fulllineitems}
\phantomsection\label{\detokenize{FLIMGraphics:FLIMGraphics.FLIMGraphic.checkDeviceStatus}}
\pysigstartsignatures
\pysiglinewithargsret{\sphinxbfcode{\sphinxupquote{checkDeviceStatus}}}{}{}
\pysigstopsignatures
\sphinxAtStartPar
Verifies the device’s operational status by attempting to read its identification.

\sphinxAtStartPar
This function performs the following actions:
\sphinxhyphen{} Attempts to read the identification from the device to ensure it is functioning correctly.
\sphinxhyphen{} If the read operation fails, it stops any running worker thread and disconnects the device from the serial port.
\sphinxhyphen{} Displays an error message indicating that the connection with the device has been lost.
\begin{quote}\begin{description}
\sphinxlineitem{Returns}
\sphinxAtStartPar
None

\end{description}\end{quote}

\end{fulllineitems}

\index{clearGraphic() (FLIMGraphics.FLIMGraphic method)@\spxentry{clearGraphic()}\spxextra{FLIMGraphics.FLIMGraphic method}}

\begin{fulllineitems}
\phantomsection\label{\detokenize{FLIMGraphics:FLIMGraphics.FLIMGraphic.clearGraphic}}
\pysigstartsignatures
\pysiglinewithargsret{\sphinxbfcode{\sphinxupquote{clearGraphic}}}{}{}
\pysigstopsignatures
\sphinxAtStartPar
Clears the measured data and time lists when a measurement is in progress 
and also clears the graphic by executing the worker’s clear function.
\begin{quote}\begin{description}
\sphinxlineitem{Returns}
\sphinxAtStartPar
None

\end{description}\end{quote}

\end{fulllineitems}

\index{connectedDevice() (FLIMGraphics.FLIMGraphic method)@\spxentry{connectedDevice()}\spxextra{FLIMGraphics.FLIMGraphic method}}

\begin{fulllineitems}
\phantomsection\label{\detokenize{FLIMGraphics:FLIMGraphics.FLIMGraphic.connectedDevice}}
\pysigstartsignatures
\pysiglinewithargsret{\sphinxbfcode{\sphinxupquote{connectedDevice}}}{\sphinxparam{\DUrole{n,n}{device}}}{}
\pysigstopsignatures
\sphinxAtStartPar
Configures the UI options when the Tempico device is connected.

\sphinxAtStartPar
This function enables and disables the appropriate buttons and starts the timer to monitor the device status.
\sphinxhyphen{} Enables the disconnect button and disables the connect button.
\sphinxhyphen{} Starts a status timer with a 500 ms interval.
\sphinxhyphen{} Assigns the connected device to the class attribute and enables the start button.
\begin{quote}\begin{description}
\sphinxlineitem{Parameters}
\sphinxAtStartPar
\sphinxstyleliteralstrong{\sphinxupquote{device}} \textendash{} The connected Tempico device.

\sphinxlineitem{Returns}
\sphinxAtStartPar
None

\end{description}\end{quote}

\end{fulllineitems}

\index{disconnectedDevice() (FLIMGraphics.FLIMGraphic method)@\spxentry{disconnectedDevice()}\spxextra{FLIMGraphics.FLIMGraphic method}}

\begin{fulllineitems}
\phantomsection\label{\detokenize{FLIMGraphics:FLIMGraphics.FLIMGraphic.disconnectedDevice}}
\pysigstartsignatures
\pysiglinewithargsret{\sphinxbfcode{\sphinxupquote{disconnectedDevice}}}{}{}
\pysigstopsignatures
\sphinxAtStartPar
Configures the UI options when the Tempico device is disconnected.

\sphinxAtStartPar
This function disables and enables the appropriate buttons and stops the timer that monitors the device status.
\sphinxhyphen{} Disables the disconnect button and enables the connect button.
\sphinxhyphen{} Stops the status timer.
\sphinxhyphen{} Disables the start button.
\begin{quote}\begin{description}
\sphinxlineitem{Returns}
\sphinxAtStartPar
None

\end{description}\end{quote}

\end{fulllineitems}

\index{double\_Exponential() (FLIMGraphics.FLIMGraphic method)@\spxentry{double\_Exponential()}\spxextra{FLIMGraphics.FLIMGraphic method}}

\begin{fulllineitems}
\phantomsection\label{\detokenize{FLIMGraphics:FLIMGraphics.FLIMGraphic.double_Exponential}}
\pysigstartsignatures
\pysiglinewithargsret{\sphinxbfcode{\sphinxupquote{double\_Exponential}}}{\sphinxparam{\DUrole{n,n}{t}}\sphinxparamcomma \sphinxparam{\DUrole{n,n}{I0}}\sphinxparamcomma \sphinxparam{\DUrole{n,n}{tau0}}\sphinxparamcomma \sphinxparam{\DUrole{n,n}{tau1}}\sphinxparamcomma \sphinxparam{\DUrole{n,n}{alpha}}}{}
\pysigstopsignatures
\sphinxAtStartPar
Computes the value of a double exponential decay function.

\sphinxAtStartPar
This function calculates the value of a double exponential decay function using the given parameters: 
time \sphinxtitleref{t}, initial value \sphinxtitleref{I0}, decay constants \sphinxtitleref{tau0} and \sphinxtitleref{tau1}, and mixing parameter \sphinxtitleref{alpha}.

\sphinxAtStartPar
Variables assignment:
\sphinxhyphen{} The computed value is returned directly and not assigned to any variable within the function.
\begin{quote}\begin{description}
\sphinxlineitem{Parameters}\begin{itemize}
\item {} 
\sphinxAtStartPar
\sphinxstyleliteralstrong{\sphinxupquote{t}} \textendash{} The time at which to evaluate the function (float).

\item {} 
\sphinxAtStartPar
\sphinxstyleliteralstrong{\sphinxupquote{I0}} \textendash{} The initial value of the exponential function (float).

\item {} 
\sphinxAtStartPar
\sphinxstyleliteralstrong{\sphinxupquote{tau0}} \textendash{} The first decay constant (float).

\item {} 
\sphinxAtStartPar
\sphinxstyleliteralstrong{\sphinxupquote{tau1}} \textendash{} The second decay constant (float).

\item {} 
\sphinxAtStartPar
\sphinxstyleliteralstrong{\sphinxupquote{alpha}} \textendash{} The mixing parameter that determines the contribution of the first decay (float).

\end{itemize}

\sphinxlineitem{Returns}
\sphinxAtStartPar
The value of the double exponential decay function at time \sphinxtitleref{t}.

\sphinxlineitem{Return type}
\sphinxAtStartPar
float

\end{description}\end{quote}

\end{fulllineitems}

\index{enableAfterFinisihThread() (FLIMGraphics.FLIMGraphic method)@\spxentry{enableAfterFinisihThread()}\spxextra{FLIMGraphics.FLIMGraphic method}}

\begin{fulllineitems}
\phantomsection\label{\detokenize{FLIMGraphics:FLIMGraphics.FLIMGraphic.enableAfterFinisihThread}}
\pysigstartsignatures
\pysiglinewithargsret{\sphinxbfcode{\sphinxupquote{enableAfterFinisihThread}}}{}{}
\pysigstopsignatures
\sphinxAtStartPar
Executes when a measurement ends, either by pressing the stop button 
or when it has naturally finished. Re\sphinxhyphen{}enables the buttons, combo boxes, 
and fields necessary for a new measurement or for saving data. 
Resets the default parameters for the settings.
\begin{quote}\begin{description}
\sphinxlineitem{Returns}
\sphinxAtStartPar
None

\end{description}\end{quote}

\end{fulllineitems}

\index{exp\_decay() (FLIMGraphics.FLIMGraphic method)@\spxentry{exp\_decay()}\spxextra{FLIMGraphics.FLIMGraphic method}}

\begin{fulllineitems}
\phantomsection\label{\detokenize{FLIMGraphics:FLIMGraphics.FLIMGraphic.exp_decay}}
\pysigstartsignatures
\pysiglinewithargsret{\sphinxbfcode{\sphinxupquote{exp\_decay}}}{\sphinxparam{\DUrole{n,n}{t}}\sphinxparamcomma \sphinxparam{\DUrole{n,n}{I0}}\sphinxparamcomma \sphinxparam{\DUrole{n,n}{tau0}}}{}
\pysigstopsignatures
\sphinxAtStartPar
Computes the value of an exponential decay function.

\sphinxAtStartPar
This function calculates the value of the exponential decay function using the given parameters for time \sphinxtitleref{t}, 
initial value \sphinxtitleref{I0}, and decay constant \sphinxtitleref{tau0}.

\sphinxAtStartPar
Variables assignment:
\sphinxhyphen{} The computed value is returned directly and not assigned to any variable within the function.
\begin{quote}\begin{description}
\sphinxlineitem{Parameters}\begin{itemize}
\item {} 
\sphinxAtStartPar
\sphinxstyleliteralstrong{\sphinxupquote{t}} \textendash{} The time at which to evaluate the function (float).

\item {} 
\sphinxAtStartPar
\sphinxstyleliteralstrong{\sphinxupquote{I0}} \textendash{} The initial value of the exponential function (float).

\item {} 
\sphinxAtStartPar
\sphinxstyleliteralstrong{\sphinxupquote{tau0}} \textendash{} The decay constant (float).

\end{itemize}

\sphinxlineitem{Returns}
\sphinxAtStartPar
The value of the exponential decay function at time \sphinxtitleref{t}.

\sphinxlineitem{Return type}
\sphinxAtStartPar
float

\end{description}\end{quote}

\end{fulllineitems}

\index{finishedThreadMeasurement() (FLIMGraphics.FLIMGraphic method)@\spxentry{finishedThreadMeasurement()}\spxextra{FLIMGraphics.FLIMGraphic method}}

\begin{fulllineitems}
\phantomsection\label{\detokenize{FLIMGraphics:FLIMGraphics.FLIMGraphic.finishedThreadMeasurement}}
\pysigstartsignatures
\pysiglinewithargsret{\sphinxbfcode{\sphinxupquote{finishedThreadMeasurement}}}{}{}
\pysigstopsignatures
\sphinxAtStartPar
Executes when the measurement thread finishes.

\sphinxAtStartPar
This function enables the UI components related to new measurements 
and resets the settings to default.
\begin{quote}\begin{description}
\sphinxlineitem{Returns}
\sphinxAtStartPar
None

\end{description}\end{quote}

\end{fulllineitems}

\index{fitDoubleExponential() (FLIMGraphics.FLIMGraphic method)@\spxentry{fitDoubleExponential()}\spxextra{FLIMGraphics.FLIMGraphic method}}

\begin{fulllineitems}
\phantomsection\label{\detokenize{FLIMGraphics:FLIMGraphics.FLIMGraphic.fitDoubleExponential}}
\pysigstartsignatures
\pysiglinewithargsret{\sphinxbfcode{\sphinxupquote{fitDoubleExponential}}}{\sphinxparam{\DUrole{n,n}{xData}}\sphinxparamcomma \sphinxparam{\DUrole{n,n}{yData}}}{}
\pysigstopsignatures
\sphinxAtStartPar
Fits a double exponential decay model to the provided data.

\sphinxAtStartPar
This function fits a double exponential decay model to the given \sphinxtitleref{xData} and \sphinxtitleref{yData} using the initial guesses for 
the parameters (I0, tau0, tau1, alpha). If the fit is successful, it updates the corresponding values for each 
parameter and plots the fitted curve. If the fit fails, a warning message is displayed to the user.

\sphinxAtStartPar
Variables assignment:
\sphinxhyphen{} \sphinxtitleref{self.FitCov}: Stores the covariance values for I0, tau0, tau1, and alpha, or “nan” if not calculable.
\sphinxhyphen{} \sphinxtitleref{self.xDataFitCopy}: Stores a copy of the \sphinxtitleref{xData} for the fitted curve.
\sphinxhyphen{} \sphinxtitleref{self.yDataFitCopy}: Stores the computed y\sphinxhyphen{}values for the fitted curve.
\sphinxhyphen{} \sphinxtitleref{self.R2}: Stores the calculated R\(\sp{\text{2}}\) value for the fit.
\sphinxhyphen{} \sphinxtitleref{self.curveFit}: Updates the plot with the fitted curve data.
\begin{quote}\begin{description}
\sphinxlineitem{Parameters}\begin{itemize}
\item {} 
\sphinxAtStartPar
\sphinxstyleliteralstrong{\sphinxupquote{xData}} \textendash{} The x\sphinxhyphen{}axis data points (array\sphinxhyphen{}like).

\item {} 
\sphinxAtStartPar
\sphinxstyleliteralstrong{\sphinxupquote{yData}} \textendash{} The y\sphinxhyphen{}axis data points (array\sphinxhyphen{}like).

\end{itemize}

\sphinxlineitem{Returns}
\sphinxAtStartPar
Tuple (I0\_opt, tau0\_opt, tau1\_opt, alpha\_opt) containing the fitted values of I0, tau0, tau1, and alpha, 
or (“Undefined”, “Undefined”, “Undefined”, “Undefined”) if fitting fails.

\sphinxlineitem{Return type}
\sphinxAtStartPar
tuple(float, float, float, float) or tuple(str, str, str, str)

\end{description}\end{quote}

\end{fulllineitems}

\index{fitExpDecay() (FLIMGraphics.FLIMGraphic method)@\spxentry{fitExpDecay()}\spxextra{FLIMGraphics.FLIMGraphic method}}

\begin{fulllineitems}
\phantomsection\label{\detokenize{FLIMGraphics:FLIMGraphics.FLIMGraphic.fitExpDecay}}
\pysigstartsignatures
\pysiglinewithargsret{\sphinxbfcode{\sphinxupquote{fitExpDecay}}}{\sphinxparam{\DUrole{n,n}{xData}}\sphinxparamcomma \sphinxparam{\DUrole{n,n}{yData}}}{}
\pysigstopsignatures
\sphinxAtStartPar
Calculates the exponential decay fit for the given data.

\sphinxAtStartPar
This function attempts to fit an exponential decay model to the provided \sphinxtitleref{xData} and \sphinxtitleref{yData}. If successful,
the optimal parameters (I0, tau0) are used to update the respective labels in the parameters table along with
their uncertainties (covariance), and the fitted curve is plotted. If the fit cannot be performed, it alerts
the user via a message box.

\sphinxAtStartPar
Variables assignment:
\sphinxhyphen{} \sphinxtitleref{self.FitCov}: Contains the covariance values for I0 and tau0, or “nan” if not calculable.
\sphinxhyphen{} \sphinxtitleref{self.xDataFitCopy}: Stores a copy of the \sphinxtitleref{xData} for the fitted curve.
\sphinxhyphen{} \sphinxtitleref{self.yDataFitCopy}: Stores the computed y\sphinxhyphen{}values for the fitted curve.
\sphinxhyphen{} \sphinxtitleref{self.R2}: Stores the calculated R\(\sp{\text{2}}\) value for the fit.
\sphinxhyphen{} \sphinxtitleref{self.curveFit}: Updates the plot with the fitted curve data.
\begin{quote}\begin{description}
\sphinxlineitem{Parameters}\begin{itemize}
\item {} 
\sphinxAtStartPar
\sphinxstyleliteralstrong{\sphinxupquote{xData}} \textendash{} The x\sphinxhyphen{}axis data points (array\sphinxhyphen{}like).

\item {} 
\sphinxAtStartPar
\sphinxstyleliteralstrong{\sphinxupquote{yData}} \textendash{} The y\sphinxhyphen{}axis data points (array\sphinxhyphen{}like).

\end{itemize}

\sphinxlineitem{Returns}
\sphinxAtStartPar
Tuple (I0\_opt, tau0\_opt) containing the fitted values of I0 and tau0, or (“Undefined”, “Undefined”) if fitting fails.

\sphinxlineitem{Return type}
\sphinxAtStartPar
tuple(float, float) or tuple(str, str)

\end{description}\end{quote}

\end{fulllineitems}

\index{fitKohlrauschFit() (FLIMGraphics.FLIMGraphic method)@\spxentry{fitKohlrauschFit()}\spxextra{FLIMGraphics.FLIMGraphic method}}

\begin{fulllineitems}
\phantomsection\label{\detokenize{FLIMGraphics:FLIMGraphics.FLIMGraphic.fitKohlrauschFit}}
\pysigstartsignatures
\pysiglinewithargsret{\sphinxbfcode{\sphinxupquote{fitKohlrauschFit}}}{\sphinxparam{\DUrole{n,n}{xData}}\sphinxparamcomma \sphinxparam{\DUrole{n,n}{yData}}}{}
\pysigstopsignatures
\sphinxAtStartPar
Calculates the Kohlrausch fit for the given data.

\sphinxAtStartPar
This function attempts to fit a Kohlrausch (stretched exponential) model to the provided \sphinxtitleref{xData} and \sphinxtitleref{yData}.
If successful, the optimal parameters (I0, tau0, β) are used to update the respective labels in the parameters
table along with their uncertainties (covariance), and the fitted curve is plotted. If the fit cannot be performed,
it alerts the user via a message box.

\sphinxAtStartPar
Variables assignment:
\sphinxhyphen{} \sphinxtitleref{self.FitCov}: Contains the covariance values for I0, tau0, and β, or “nan” if not calculable.
\sphinxhyphen{} \sphinxtitleref{self.xDataFitCopy}: Stores a copy of the \sphinxtitleref{xData} for the fitted curve.
\sphinxhyphen{} \sphinxtitleref{self.yDataFitCopy}: Stores the computed y\sphinxhyphen{}values for the fitted curve.
\sphinxhyphen{} \sphinxtitleref{self.R2}: Stores the calculated R\(\sp{\text{2}}\) value for the fit.
\sphinxhyphen{} \sphinxtitleref{self.curveFit}: Updates the plot with the fitted curve data.
\begin{quote}\begin{description}
\sphinxlineitem{Parameters}\begin{itemize}
\item {} 
\sphinxAtStartPar
\sphinxstyleliteralstrong{\sphinxupquote{xData}} \textendash{} The x\sphinxhyphen{}axis data points (array\sphinxhyphen{}like).

\item {} 
\sphinxAtStartPar
\sphinxstyleliteralstrong{\sphinxupquote{yData}} \textendash{} The y\sphinxhyphen{}axis data points (array\sphinxhyphen{}like).

\end{itemize}

\sphinxlineitem{Returns}
\sphinxAtStartPar
Tuple (I0\_opt, tau0\_opt, beta\_opt) containing the fitted values of I0, tau0, and β, or (“Undefined”, “Undefined”, “Undefined”) if fitting fails.

\sphinxlineitem{Return type}
\sphinxAtStartPar
tuple(float, float, float) or tuple(str, str, str)

\end{description}\end{quote}

\end{fulllineitems}

\index{fitShiiftedExponential() (FLIMGraphics.FLIMGraphic method)@\spxentry{fitShiiftedExponential()}\spxextra{FLIMGraphics.FLIMGraphic method}}

\begin{fulllineitems}
\phantomsection\label{\detokenize{FLIMGraphics:FLIMGraphics.FLIMGraphic.fitShiiftedExponential}}
\pysigstartsignatures
\pysiglinewithargsret{\sphinxbfcode{\sphinxupquote{fitShiiftedExponential}}}{\sphinxparam{\DUrole{n,n}{xData}}\sphinxparamcomma \sphinxparam{\DUrole{n,n}{yData}}}{}
\pysigstopsignatures
\sphinxAtStartPar
Fits a shifted exponential decay model to the provided data.

\sphinxAtStartPar
This function fits a shifted exponential decay model to the given \sphinxtitleref{xData} and \sphinxtitleref{yData} using the initial guesses for 
the parameters (I0, tau0, alpha, b). If the fit is successful, it updates the corresponding values for each parameter 
and plots the fitted curve. If the fit fails, a warning message is displayed to the user.

\sphinxAtStartPar
Variables assignment:
\sphinxhyphen{} \sphinxtitleref{self.FitCov}: Stores the covariance values for I0, tau0, alpha, and b, or “nan” if not calculable.
\sphinxhyphen{} \sphinxtitleref{self.xDataFitCopy}: Stores a copy of the \sphinxtitleref{xData} for the fitted curve.
\sphinxhyphen{} \sphinxtitleref{self.yDataFitCopy}: Stores the computed y\sphinxhyphen{}values for the fitted curve.
\sphinxhyphen{} \sphinxtitleref{self.R2}: Stores the calculated R\(\sp{\text{2}}\) value for the fit.
\sphinxhyphen{} \sphinxtitleref{self.curveFit}: Updates the plot with the fitted curve data.
\begin{quote}\begin{description}
\sphinxlineitem{Parameters}\begin{itemize}
\item {} 
\sphinxAtStartPar
\sphinxstyleliteralstrong{\sphinxupquote{xData}} \textendash{} The x\sphinxhyphen{}axis data points (array\sphinxhyphen{}like).

\item {} 
\sphinxAtStartPar
\sphinxstyleliteralstrong{\sphinxupquote{yData}} \textendash{} The y\sphinxhyphen{}axis data points (array\sphinxhyphen{}like).

\end{itemize}

\sphinxlineitem{Returns}
\sphinxAtStartPar
Tuple (I0\_opt, tau0\_opt, alpha\_opt, b\_opt) containing the fitted values of I0, tau0, alpha, and b, or 
(“Undefined”, “Undefined”, “Undefined”, “Undefined”) if fitting fails.

\sphinxlineitem{Return type}
\sphinxAtStartPar
tuple(float, float, float, float) or tuple(str, str, str, str)

\end{description}\end{quote}

\end{fulllineitems}

\index{getTempicoChannel() (FLIMGraphics.FLIMGraphic method)@\spxentry{getTempicoChannel()}\spxextra{FLIMGraphics.FLIMGraphic method}}

\begin{fulllineitems}
\phantomsection\label{\detokenize{FLIMGraphics:FLIMGraphics.FLIMGraphic.getTempicoChannel}}
\pysigstartsignatures
\pysiglinewithargsret{\sphinxbfcode{\sphinxupquote{getTempicoChannel}}}{}{}
\pysigstopsignatures
\sphinxAtStartPar
Assigns the Tempico device channels based on the selected values from the combo boxes.
The selected start and stop channels are assigned to the variables \sphinxtitleref{currentStartChannel} 
and \sphinxtitleref{currentStopChannel}, respectively. It also enables the corresponding channels.
\begin{quote}\begin{description}
\sphinxlineitem{Returns}
\sphinxAtStartPar
None

\end{description}\end{quote}

\end{fulllineitems}

\index{getUnits() (FLIMGraphics.FLIMGraphic method)@\spxentry{getUnits()}\spxextra{FLIMGraphics.FLIMGraphic method}}

\begin{fulllineitems}
\phantomsection\label{\detokenize{FLIMGraphics:FLIMGraphics.FLIMGraphic.getUnits}}
\pysigstartsignatures
\pysiglinewithargsret{\sphinxbfcode{\sphinxupquote{getUnits}}}{\sphinxparam{\DUrole{n,n}{value}}}{}
\pysigstopsignatures
\sphinxAtStartPar
Receives a numerical value in picoseconds (ps) and returns a list with two values:
the appropriate units for the value and a number indicating by how much to divide 
the value for conversion.
\begin{quote}\begin{description}
\sphinxlineitem{Parameters}
\sphinxAtStartPar
\sphinxstyleliteralstrong{\sphinxupquote{value}} \textendash{} The numerical value in picoseconds to convert.

\sphinxlineitem{Returns}
\sphinxAtStartPar
A list containing the appropriate unit as a string and a number indicating 
the divisor for conversion.

\end{description}\end{quote}

\end{fulllineitems}

\index{indexChangeStartChannel() (FLIMGraphics.FLIMGraphic method)@\spxentry{indexChangeStartChannel()}\spxextra{FLIMGraphics.FLIMGraphic method}}

\begin{fulllineitems}
\phantomsection\label{\detokenize{FLIMGraphics:FLIMGraphics.FLIMGraphic.indexChangeStartChannel}}
\pysigstartsignatures
\pysiglinewithargsret{\sphinxbfcode{\sphinxupquote{indexChangeStartChannel}}}{}{}
\pysigstopsignatures
\sphinxAtStartPar
Verifies that the options of the start and stop combo boxes are not the same 
when the start combo box is changed. If they are the same, it reverts to 
the previously selected option.
\begin{quote}\begin{description}
\sphinxlineitem{Returns}
\sphinxAtStartPar
None

\end{description}\end{quote}

\end{fulllineitems}

\index{indexChangeStopChannel() (FLIMGraphics.FLIMGraphic method)@\spxentry{indexChangeStopChannel()}\spxextra{FLIMGraphics.FLIMGraphic method}}

\begin{fulllineitems}
\phantomsection\label{\detokenize{FLIMGraphics:FLIMGraphics.FLIMGraphic.indexChangeStopChannel}}
\pysigstartsignatures
\pysiglinewithargsret{\sphinxbfcode{\sphinxupquote{indexChangeStopChannel}}}{}{}
\pysigstopsignatures
\sphinxAtStartPar
Verifies that the options of the start and stop combo boxes are not the same 
when the stop combo box is changed. If they are the same, it reverts to 
the previously selected option.
\begin{quote}\begin{description}
\sphinxlineitem{Returns}
\sphinxAtStartPar
None

\end{description}\end{quote}

\end{fulllineitems}

\index{initialParametersDialog() (FLIMGraphics.FLIMGraphic method)@\spxentry{initialParametersDialog()}\spxextra{FLIMGraphics.FLIMGraphic method}}

\begin{fulllineitems}
\phantomsection\label{\detokenize{FLIMGraphics:FLIMGraphics.FLIMGraphic.initialParametersDialog}}
\pysigstartsignatures
\pysiglinewithargsret{\sphinxbfcode{\sphinxupquote{initialParametersDialog}}}{}{}
\pysigstopsignatures
\sphinxAtStartPar
Displays a dialog for selecting initial parameters for fitting functions.

\sphinxAtStartPar
This dialog allows the user to select a type of fitting function and input the corresponding parameters.
The parameters that can be set include:
\sphinxhyphen{} I0 (initial value)
\sphinxhyphen{} tau0 (decay constant)
\sphinxhyphen{} beta (for Kohlrausch fit)
\sphinxhyphen{} alpha (for Shifted Exponential fit)
\sphinxhyphen{} b (for Shifted Exponential fit)

\sphinxAtStartPar
The dialog contains a ComboBox to select the function type, and the input fields dynamically update based on
the selected function. It includes “Apply” and “Reset” buttons to apply changes or revert to default values.

\sphinxAtStartPar
Variables assignment:
\sphinxhyphen{} The input values from the dialog are assigned to respective attributes, but are not directly returned.
\begin{quote}\begin{description}
\sphinxlineitem{Returns}
\sphinxAtStartPar
None

\end{description}\end{quote}

\end{fulllineitems}

\index{insertParameters() (FLIMGraphics.FLIMGraphic method)@\spxentry{insertParameters()}\spxextra{FLIMGraphics.FLIMGraphic method}}

\begin{fulllineitems}
\phantomsection\label{\detokenize{FLIMGraphics:FLIMGraphics.FLIMGraphic.insertParameters}}
\pysigstartsignatures
\pysiglinewithargsret{\sphinxbfcode{\sphinxupquote{insertParameters}}}{\sphinxparam{\DUrole{n,n}{index}}\sphinxparamcomma \sphinxparam{\DUrole{n,n}{Parameter}}\sphinxparamcomma \sphinxparam{\DUrole{n,n}{Value}}\sphinxparamcomma \sphinxparam{\DUrole{n,n}{Cov}}\sphinxparamcomma \sphinxparam{\DUrole{n,n}{Units}}}{}
\pysigstopsignatures
\sphinxAtStartPar
Inserts the parameter values into the fitting table at the specified row.

\sphinxAtStartPar
This function updates the table used for displaying the fitted parameters by inserting the parameter name, its 
value, the covariance (uncertainty), and the units in the specified row of the table.

\sphinxAtStartPar
Variables assignment:
\sphinxhyphen{} The parameter name is assigned to the first column of the specified row (\sphinxtitleref{index}).
\sphinxhyphen{} The value of the parameter is assigned to the second column.
\sphinxhyphen{} The covariance of the parameter is assigned to the third column.
\sphinxhyphen{} The units of the parameter are assigned to the fourth column.
\begin{quote}\begin{description}
\sphinxlineitem{Parameters}\begin{itemize}
\item {} 
\sphinxAtStartPar
\sphinxstyleliteralstrong{\sphinxupquote{index}} \textendash{} The row index where the values will be inserted (int).

\item {} 
\sphinxAtStartPar
\sphinxstyleliteralstrong{\sphinxupquote{Parameter}} \textendash{} The name of the parameter to be displayed (str).

\item {} 
\sphinxAtStartPar
\sphinxstyleliteralstrong{\sphinxupquote{Value}} \textendash{} The value of the parameter to be displayed (str).

\item {} 
\sphinxAtStartPar
\sphinxstyleliteralstrong{\sphinxupquote{Cov}} \textendash{} The covariance (uncertainty) of the parameter (str).

\item {} 
\sphinxAtStartPar
\sphinxstyleliteralstrong{\sphinxupquote{Units}} \textendash{} The units of the parameter (str).

\end{itemize}

\sphinxlineitem{Returns}
\sphinxAtStartPar
None

\end{description}\end{quote}

\end{fulllineitems}

\index{kohl\_decay() (FLIMGraphics.FLIMGraphic method)@\spxentry{kohl\_decay()}\spxextra{FLIMGraphics.FLIMGraphic method}}

\begin{fulllineitems}
\phantomsection\label{\detokenize{FLIMGraphics:FLIMGraphics.FLIMGraphic.kohl_decay}}
\pysigstartsignatures
\pysiglinewithargsret{\sphinxbfcode{\sphinxupquote{kohl\_decay}}}{\sphinxparam{\DUrole{n,n}{t}}\sphinxparamcomma \sphinxparam{\DUrole{n,n}{I0}}\sphinxparamcomma \sphinxparam{\DUrole{n,n}{tau0}}\sphinxparamcomma \sphinxparam{\DUrole{n,n}{beta}}}{}
\pysigstopsignatures
\sphinxAtStartPar
Computes the value of a Kohlrausch (stretched exponential) decay function.

\sphinxAtStartPar
This function calculates the value of the stretched exponential (Kohlrausch) decay function using the given parameters: 
time \sphinxtitleref{t}, initial value \sphinxtitleref{I0}, decay constant \sphinxtitleref{tau0}, and stretching parameter \sphinxtitleref{beta}.

\sphinxAtStartPar
Variables assignment:
\sphinxhyphen{} The computed value is returned directly and not assigned to any variable within the function.
\begin{quote}\begin{description}
\sphinxlineitem{Parameters}\begin{itemize}
\item {} 
\sphinxAtStartPar
\sphinxstyleliteralstrong{\sphinxupquote{t}} \textendash{} The time at which to evaluate the function (float).

\item {} 
\sphinxAtStartPar
\sphinxstyleliteralstrong{\sphinxupquote{I0}} \textendash{} The initial value of the exponential function (float).

\item {} 
\sphinxAtStartPar
\sphinxstyleliteralstrong{\sphinxupquote{tau0}} \textendash{} The decay constant (float).

\item {} 
\sphinxAtStartPar
\sphinxstyleliteralstrong{\sphinxupquote{beta}} \textendash{} The stretching exponent, controlling the deviation from a simple exponential decay (float).

\end{itemize}

\sphinxlineitem{Returns}
\sphinxAtStartPar
The value of the stretched exponential decay function at time \sphinxtitleref{t}.

\sphinxlineitem{Return type}
\sphinxAtStartPar
float

\end{description}\end{quote}

\end{fulllineitems}

\index{maxRound() (FLIMGraphics.FLIMGraphic method)@\spxentry{maxRound()}\spxextra{FLIMGraphics.FLIMGraphic method}}

\begin{fulllineitems}
\phantomsection\label{\detokenize{FLIMGraphics:FLIMGraphics.FLIMGraphic.maxRound}}
\pysigstartsignatures
\pysiglinewithargsret{\sphinxbfcode{\sphinxupquote{maxRound}}}{\sphinxparam{\DUrole{n,n}{string}}}{}
\pysigstopsignatures
\sphinxAtStartPar
Determines the number of decimal places needed for at least three significant figures.
\begin{quote}\begin{description}
\sphinxlineitem{Parameters}
\sphinxAtStartPar
\sphinxstyleliteralstrong{\sphinxupquote{string}} \textendash{} The number as a string (str).

\sphinxlineitem{Returns}
\sphinxAtStartPar
Number of decimal places (int).

\end{description}\end{quote}

\end{fulllineitems}

\index{resetInitialDialog() (FLIMGraphics.FLIMGraphic method)@\spxentry{resetInitialDialog()}\spxextra{FLIMGraphics.FLIMGraphic method}}

\begin{fulllineitems}
\phantomsection\label{\detokenize{FLIMGraphics:FLIMGraphics.FLIMGraphic.resetInitialDialog}}
\pysigstartsignatures
\pysiglinewithargsret{\sphinxbfcode{\sphinxupquote{resetInitialDialog}}}{}{}
\pysigstopsignatures
\sphinxAtStartPar
Resets the initial parameters in the dialog based on the measured data.

\sphinxAtStartPar
This function is called when the user clicks the “Default Values” button in the parameter dialog. 
It resets the initial parameters to sensible defaults derived from the measured data. 
The following resets occur based on the selected fitting function:
\begin{itemize}
\item {} \begin{description}
\sphinxlineitem{For Exponential fit:}\begin{itemize}
\item {} 
\sphinxAtStartPar
I0 is set to the maximum value of measured data.

\item {} 
\sphinxAtStartPar
tau0 is set to the mean of measured time.

\end{itemize}

\end{description}

\item {} \begin{description}
\sphinxlineitem{For Kohlrausch fit:}\begin{itemize}
\item {} 
\sphinxAtStartPar
I0 is set to the maximum value of measured data.

\item {} 
\sphinxAtStartPar
tau0 is set to the mean of measured time.

\item {} 
\sphinxAtStartPar
beta is set to 1.0 (default value).

\end{itemize}

\end{description}

\item {} \begin{description}
\sphinxlineitem{For Shifted Exponential fit:}\begin{itemize}
\item {} 
\sphinxAtStartPar
I0 is set to the maximum value of measured data.

\item {} 
\sphinxAtStartPar
tau0 is set to the mean of measured time.

\item {} 
\sphinxAtStartPar
alpha is set to 0 (default value).

\item {} 
\sphinxAtStartPar
b is set to 0 (default value).

\end{itemize}

\end{description}

\item {} \begin{description}
\sphinxlineitem{For Double Exponential fit:}\begin{itemize}
\item {} 
\sphinxAtStartPar
I0 is set to the maximum value of measured data.

\item {} 
\sphinxAtStartPar
tau0 is set to the mean of measured time.

\item {} 
\sphinxAtStartPar
alpha is set to 0 (default value).

\item {} 
\sphinxAtStartPar
tau1 is set to the mean of measured time.

\end{itemize}

\end{description}

\end{itemize}

\sphinxAtStartPar
The corresponding input fields in the dialog are updated with these new values. 
Flags indicating that the initial parameters have not changed are also reset to False for the relevant fit type.
\begin{quote}\begin{description}
\sphinxlineitem{Returns}
\sphinxAtStartPar
None

\end{description}\end{quote}

\end{fulllineitems}

\index{resetParametersLabels() (FLIMGraphics.FLIMGraphic method)@\spxentry{resetParametersLabels()}\spxextra{FLIMGraphics.FLIMGraphic method}}

\begin{fulllineitems}
\phantomsection\label{\detokenize{FLIMGraphics:FLIMGraphics.FLIMGraphic.resetParametersLabels}}
\pysigstartsignatures
\pysiglinewithargsret{\sphinxbfcode{\sphinxupquote{resetParametersLabels}}}{}{}
\pysigstopsignatures
\sphinxAtStartPar
Resets all the fit parameters to ‘Undefined’ and clears the current fit type.

\sphinxAtStartPar
This function sets the list \sphinxtitleref{FitParameters} to contain “Undefined” values for all parameters and clears \sphinxtitleref{currentFit}.
Additionally, it resets the \sphinxtitleref{R2} value to “Undefined”. Depending on the selected fit type in \sphinxtitleref{functionComboBox},
the respective parameter labels in the table are updated to reflect “Undefined” values.

\sphinxAtStartPar
Variables assignment:
\sphinxhyphen{} \sphinxtitleref{FitParameters} is reset to a list of “Undefined”.
\sphinxhyphen{} \sphinxtitleref{currentFit} is cleared to an empty string.
\sphinxhyphen{} \sphinxtitleref{R2} is reset to “Undefined”.
\sphinxhyphen{} The \sphinxtitleref{insertParameters} method is used to update the table with “Undefined” values for each parameter
based on the selected function in \sphinxtitleref{functionComboBox}.
\begin{quote}\begin{description}
\sphinxlineitem{Returns}
\sphinxAtStartPar
None

\end{description}\end{quote}

\end{fulllineitems}

\index{roundStringPCov() (FLIMGraphics.FLIMGraphic method)@\spxentry{roundStringPCov()}\spxextra{FLIMGraphics.FLIMGraphic method}}

\begin{fulllineitems}
\phantomsection\label{\detokenize{FLIMGraphics:FLIMGraphics.FLIMGraphic.roundStringPCov}}
\pysigstartsignatures
\pysiglinewithargsret{\sphinxbfcode{\sphinxupquote{roundStringPCov}}}{\sphinxparam{\DUrole{n,n}{number}}}{}
\pysigstopsignatures
\sphinxAtStartPar
Converts a number to a string representation in scientific notation with two decimal places.
\begin{quote}\begin{description}
\sphinxlineitem{Parameters}
\sphinxAtStartPar
\sphinxstyleliteralstrong{\sphinxupquote{number}} \textendash{} The input number (float).

\sphinxlineitem{Returns}
\sphinxAtStartPar
String representation in scientific notation (str).

\end{description}\end{quote}

\end{fulllineitems}

\index{saveFLIMData() (FLIMGraphics.FLIMGraphic method)@\spxentry{saveFLIMData()}\spxextra{FLIMGraphics.FLIMGraphic method}}

\begin{fulllineitems}
\phantomsection\label{\detokenize{FLIMGraphics:FLIMGraphics.FLIMGraphic.saveFLIMData}}
\pysigstartsignatures
\pysiglinewithargsret{\sphinxbfcode{\sphinxupquote{saveFLIMData}}}{}{}
\pysigstopsignatures
\sphinxAtStartPar
Saves the data in the selected format (TXT, CSV, or DAT) based on the user’s choice.
\begin{quote}\begin{description}
\sphinxlineitem{Returns}
\sphinxAtStartPar
None

\end{description}\end{quote}

\end{fulllineitems}

\index{saveMode() (FLIMGraphics.FLIMGraphic method)@\spxentry{saveMode()}\spxextra{FLIMGraphics.FLIMGraphic method}}

\begin{fulllineitems}
\phantomsection\label{\detokenize{FLIMGraphics:FLIMGraphics.FLIMGraphic.saveMode}}
\pysigstartsignatures
\pysiglinewithargsret{\sphinxbfcode{\sphinxupquote{saveMode}}}{}{}
\pysigstopsignatures
\sphinxAtStartPar
Saves the current modes of the channels before starting a measurement.
The modes are stored in the variables \sphinxtitleref{oldChannelA}, \sphinxtitleref{oldChannelB}, 
\sphinxtitleref{oldChannelC}, and \sphinxtitleref{oldChannelD}, allowing for restoration to the original settings after the measurement.
\begin{quote}\begin{description}
\sphinxlineitem{Returns}
\sphinxAtStartPar
None

\end{description}\end{quote}

\end{fulllineitems}

\index{savePlotFLIM() (FLIMGraphics.FLIMGraphic method)@\spxentry{savePlotFLIM()}\spxextra{FLIMGraphics.FLIMGraphic method}}

\begin{fulllineitems}
\phantomsection\label{\detokenize{FLIMGraphics:FLIMGraphics.FLIMGraphic.savePlotFLIM}}
\pysigstartsignatures
\pysiglinewithargsret{\sphinxbfcode{\sphinxupquote{savePlotFLIM}}}{}{}
\pysigstopsignatures
\sphinxAtStartPar
Saves the graph image in the specified format (PNG, TIFF, or JPG) based on the user’s selection.
:return: None

\end{fulllineitems}

\index{setOldMode() (FLIMGraphics.FLIMGraphic method)@\spxentry{setOldMode()}\spxextra{FLIMGraphics.FLIMGraphic method}}

\begin{fulllineitems}
\phantomsection\label{\detokenize{FLIMGraphics:FLIMGraphics.FLIMGraphic.setOldMode}}
\pysigstartsignatures
\pysiglinewithargsret{\sphinxbfcode{\sphinxupquote{setOldMode}}}{}{}
\pysigstopsignatures
\sphinxAtStartPar
Restores the modes of the channels after a measurement is completed.
The channels are set to their previous modes stored in \sphinxtitleref{oldChannelA}, 
\sphinxtitleref{oldChannelB}, \sphinxtitleref{oldChannelC}, and \sphinxtitleref{oldChannelD}.
\begin{quote}\begin{description}
\sphinxlineitem{Returns}
\sphinxAtStartPar
None

\end{description}\end{quote}

\end{fulllineitems}

\index{shifted\_decay\_function() (FLIMGraphics.FLIMGraphic method)@\spxentry{shifted\_decay\_function()}\spxextra{FLIMGraphics.FLIMGraphic method}}

\begin{fulllineitems}
\phantomsection\label{\detokenize{FLIMGraphics:FLIMGraphics.FLIMGraphic.shifted_decay_function}}
\pysigstartsignatures
\pysiglinewithargsret{\sphinxbfcode{\sphinxupquote{shifted\_decay\_function}}}{\sphinxparam{\DUrole{n,n}{t}}\sphinxparamcomma \sphinxparam{\DUrole{n,n}{I0}}\sphinxparamcomma \sphinxparam{\DUrole{n,n}{tau0}}\sphinxparamcomma \sphinxparam{\DUrole{n,n}{alpha}}\sphinxparamcomma \sphinxparam{\DUrole{n,n}{b}}}{}
\pysigstopsignatures
\sphinxAtStartPar
Computes the value of a shifted exponential decay function.

\sphinxAtStartPar
This function calculates the value of a shifted exponential decay function using the given parameters: 
time \sphinxtitleref{t}, initial value \sphinxtitleref{I0}, decay constant \sphinxtitleref{tau0}, shift parameter \sphinxtitleref{alpha}, and constant offset \sphinxtitleref{b}.

\sphinxAtStartPar
Variables assignment:
\sphinxhyphen{} The computed value is returned directly and not assigned to any variable within the function.
\begin{quote}\begin{description}
\sphinxlineitem{Parameters}\begin{itemize}
\item {} 
\sphinxAtStartPar
\sphinxstyleliteralstrong{\sphinxupquote{t}} \textendash{} The time at which to evaluate the function (float).

\item {} 
\sphinxAtStartPar
\sphinxstyleliteralstrong{\sphinxupquote{I0}} \textendash{} The initial value of the exponential function (float).

\item {} 
\sphinxAtStartPar
\sphinxstyleliteralstrong{\sphinxupquote{tau0}} \textendash{} The decay constant (float).

\item {} 
\sphinxAtStartPar
\sphinxstyleliteralstrong{\sphinxupquote{alpha}} \textendash{} The shift parameter, determining the horizontal shift of the decay function (float).

\item {} 
\sphinxAtStartPar
\sphinxstyleliteralstrong{\sphinxupquote{b}} \textendash{} The constant offset added to the decay function (float).

\end{itemize}

\sphinxlineitem{Returns}
\sphinxAtStartPar
The value of the shifted exponential decay function at time \sphinxtitleref{t}.

\sphinxlineitem{Return type}
\sphinxAtStartPar
float

\end{description}\end{quote}

\end{fulllineitems}

\index{startMeasurement() (FLIMGraphics.FLIMGraphic method)@\spxentry{startMeasurement()}\spxextra{FLIMGraphics.FLIMGraphic method}}

\begin{fulllineitems}
\phantomsection\label{\detokenize{FLIMGraphics:FLIMGraphics.FLIMGraphic.startMeasurement}}
\pysigstartsignatures
\pysiglinewithargsret{\sphinxbfcode{\sphinxupquote{startMeasurement}}}{}{}
\pysigstopsignatures
\sphinxAtStartPar
Initiates the measurement process by clearing previous data, resetting the UI, and starting a new measurement thread.

\sphinxAtStartPar
This function performs several actions to prepare for and start a new measurement:
\sphinxhyphen{} Stops the status timer and disables/enables relevant buttons.
\sphinxhyphen{} Clears previous graph data and resets parameter labels.
\sphinxhyphen{} Saves the current channel modes and resets save parameters.
\sphinxhyphen{} Updates the status label and color to indicate a running measurement.
\sphinxhyphen{} Clears previously measured data and time, resetting the plot.
\sphinxhyphen{} Retrieves the selected channels and initializes a new measurement thread with the selected parameters.
\sphinxhyphen{} Connects signals from the worker thread to corresponding slots in the main thread for UI updates.
\begin{quote}\begin{description}
\sphinxlineitem{Returns}
\sphinxAtStartPar
None

\end{description}\end{quote}

\end{fulllineitems}

\index{startTimer() (FLIMGraphics.FLIMGraphic method)@\spxentry{startTimer()}\spxextra{FLIMGraphics.FLIMGraphic method}}

\begin{fulllineitems}
\phantomsection\label{\detokenize{FLIMGraphics:FLIMGraphics.FLIMGraphic.startTimer}}
\pysigstartsignatures
\pysiglinewithargsret{\sphinxbfcode{\sphinxupquote{startTimer}}}{}{}
\pysigstopsignatures
\sphinxAtStartPar
Starts the timer that triggers the update\_timer function every second.

\sphinxAtStartPar
This function initiates the timer for the ongoing measurement, allowing
the update\_timer function to be executed at one\sphinxhyphen{}second intervals.
\begin{quote}\begin{description}
\sphinxlineitem{Returns}
\sphinxAtStartPar
None

\end{description}\end{quote}

\end{fulllineitems}

\index{stopMeasurement() (FLIMGraphics.FLIMGraphic method)@\spxentry{stopMeasurement()}\spxextra{FLIMGraphics.FLIMGraphic method}}

\begin{fulllineitems}
\phantomsection\label{\detokenize{FLIMGraphics:FLIMGraphics.FLIMGraphic.stopMeasurement}}
\pysigstartsignatures
\pysiglinewithargsret{\sphinxbfcode{\sphinxupquote{stopMeasurement}}}{}{}
\pysigstopsignatures
\sphinxAtStartPar
Stops the measurement thread if it is created and executes the 
function enableAfterFinishedThread() if the thread is not created.
\begin{quote}\begin{description}
\sphinxlineitem{Returns}
\sphinxAtStartPar
None

\end{description}\end{quote}

\end{fulllineitems}

\index{stopTimer() (FLIMGraphics.FLIMGraphic method)@\spxentry{stopTimer()}\spxextra{FLIMGraphics.FLIMGraphic method}}

\begin{fulllineitems}
\phantomsection\label{\detokenize{FLIMGraphics:FLIMGraphics.FLIMGraphic.stopTimer}}
\pysigstartsignatures
\pysiglinewithargsret{\sphinxbfcode{\sphinxupquote{stopTimer}}}{}{}
\pysigstopsignatures
\sphinxAtStartPar
Stops the timer that triggers the update\_timer function.

\sphinxAtStartPar
This function halts the timer responsible for updating the measurement time
and sets the totalTime label to indicate that no measurement is currently running.
\begin{quote}\begin{description}
\sphinxlineitem{Returns}
\sphinxAtStartPar
None

\end{description}\end{quote}

\end{fulllineitems}

\index{timeRangeValue() (FLIMGraphics.FLIMGraphic method)@\spxentry{timeRangeValue()}\spxextra{FLIMGraphics.FLIMGraphic method}}

\begin{fulllineitems}
\phantomsection\label{\detokenize{FLIMGraphics:FLIMGraphics.FLIMGraphic.timeRangeValue}}
\pysigstartsignatures
\pysiglinewithargsret{\sphinxbfcode{\sphinxupquote{timeRangeValue}}}{}{}
\pysigstopsignatures
\sphinxAtStartPar
Converts the value from the timeRange label to picoseconds (ps).
\begin{quote}\begin{description}
\sphinxlineitem{Returns}
\sphinxAtStartPar
float
The value converted to picoseconds.

\end{description}\end{quote}

\end{fulllineitems}

\index{updateLabel() (FLIMGraphics.FLIMGraphic method)@\spxentry{updateLabel()}\spxextra{FLIMGraphics.FLIMGraphic method}}

\begin{fulllineitems}
\phantomsection\label{\detokenize{FLIMGraphics:FLIMGraphics.FLIMGraphic.updateLabel}}
\pysigstartsignatures
\pysiglinewithargsret{\sphinxbfcode{\sphinxupquote{updateLabel}}}{\sphinxparam{\DUrole{n,n}{units}}}{}
\pysigstopsignatures
\sphinxAtStartPar
Updates the label of the x\sphinxhyphen{}axis in the graph with the correct units for the data.

\sphinxAtStartPar
This function takes the units as a string and updates the x\sphinxhyphen{}axis label accordingly.
\begin{quote}\begin{description}
\sphinxlineitem{Parameters}
\sphinxAtStartPar
\sphinxstyleliteralstrong{\sphinxupquote{units}} \textendash{} The units to be displayed on the x\sphinxhyphen{}axis label (str).

\sphinxlineitem{Returns}
\sphinxAtStartPar
None

\end{description}\end{quote}

\end{fulllineitems}

\index{updateLabels() (FLIMGraphics.FLIMGraphic method)@\spxentry{updateLabels()}\spxextra{FLIMGraphics.FLIMGraphic method}}

\begin{fulllineitems}
\phantomsection\label{\detokenize{FLIMGraphics:FLIMGraphics.FLIMGraphic.updateLabels}}
\pysigstartsignatures
\pysiglinewithargsret{\sphinxbfcode{\sphinxupquote{updateLabels}}}{\sphinxparam{\DUrole{n,n}{totalMeasurements}}\sphinxparamcomma \sphinxparam{\DUrole{n,n}{totalStarts}}}{}
\pysigstopsignatures
\sphinxAtStartPar
Updates the labels for total measurements and total starts with the given values.

\sphinxAtStartPar
This function sets the text of the totalMeasurements and totalStart labels 
according to the parameters provided.
\begin{quote}\begin{description}
\sphinxlineitem{Parameters}\begin{itemize}
\item {} 
\sphinxAtStartPar
\sphinxstyleliteralstrong{\sphinxupquote{totalMeasurements}} \textendash{} The text to set for the total measurements label (str).

\item {} 
\sphinxAtStartPar
\sphinxstyleliteralstrong{\sphinxupquote{totalStarts}} \textendash{} The text to set for the total starts label (str).

\end{itemize}

\sphinxlineitem{Returns}
\sphinxAtStartPar
None

\end{description}\end{quote}

\end{fulllineitems}

\index{updateMeasurement() (FLIMGraphics.FLIMGraphic method)@\spxentry{updateMeasurement()}\spxextra{FLIMGraphics.FLIMGraphic method}}

\begin{fulllineitems}
\phantomsection\label{\detokenize{FLIMGraphics:FLIMGraphics.FLIMGraphic.updateMeasurement}}
\pysigstartsignatures
\pysiglinewithargsret{\sphinxbfcode{\sphinxupquote{updateMeasurement}}}{\sphinxparam{\DUrole{n,n}{listOfNewValues}}\sphinxparamcomma \sphinxparam{\DUrole{n,n}{domainMeasurement}}}{}
\pysigstopsignatures
\sphinxAtStartPar
Updates the measured data and time with new values and refreshes the graph.

\sphinxAtStartPar
This function receives a new list of values representing counts and their corresponding
update times, assigning them to the variables measuredData and measuredTime, and 
updates the graph accordingly.
\begin{quote}\begin{description}
\sphinxlineitem{Parameters}\begin{itemize}
\item {} 
\sphinxAtStartPar
\sphinxstyleliteralstrong{\sphinxupquote{listOfNewValues}} \textendash{} The new list of measurement values (list).

\item {} 
\sphinxAtStartPar
\sphinxstyleliteralstrong{\sphinxupquote{domainMeasurement}} \textendash{} The list of times corresponding to the measurement values (list).

\end{itemize}

\sphinxlineitem{Returns}
\sphinxAtStartPar
None

\end{description}\end{quote}

\end{fulllineitems}

\index{update\_timer() (FLIMGraphics.FLIMGraphic method)@\spxentry{update\_timer()}\spxextra{FLIMGraphics.FLIMGraphic method}}

\begin{fulllineitems}
\phantomsection\label{\detokenize{FLIMGraphics:FLIMGraphics.FLIMGraphic.update_timer}}
\pysigstartsignatures
\pysiglinewithargsret{\sphinxbfcode{\sphinxupquote{update\_timer}}}{}{}
\pysigstopsignatures
\sphinxAtStartPar
Updates the timer for the ongoing measurement by adding one second.

\sphinxAtStartPar
This function increments the time variable by one second and updates the 
totalTime label to reflect the new time.
\begin{quote}\begin{description}
\sphinxlineitem{Returns}
\sphinxAtStartPar
None

\end{description}\end{quote}

\end{fulllineitems}


\end{fulllineitems}

\index{WorkerThreadFLIM (class in FLIMGraphics)@\spxentry{WorkerThreadFLIM}\spxextra{class in FLIMGraphics}}

\begin{fulllineitems}
\phantomsection\label{\detokenize{FLIMGraphics:FLIMGraphics.WorkerThreadFLIM}}
\pysigstartsignatures
\pysiglinewithargsret{\sphinxbfcode{\sphinxupquote{class\DUrole{w,w}{  }}}\sphinxcode{\sphinxupquote{FLIMGraphics.}}\sphinxbfcode{\sphinxupquote{WorkerThreadFLIM}}}{\sphinxparam{\DUrole{n,n}{deviceStartChannel}}\sphinxparamcomma \sphinxparam{\DUrole{n,n}{deviceStopChannel}}\sphinxparamcomma \sphinxparam{\DUrole{n,n}{binwidthText}}\sphinxparamcomma \sphinxparam{\DUrole{n,n}{numberMeasurements}}\sphinxparamcomma \sphinxparam{\DUrole{n,n}{device}}\sphinxparamcomma \sphinxparam{\DUrole{n,n}{TimeRange}}}{}
\pysigstopsignatures
\sphinxAtStartPar
Bases: \sphinxcode{\sphinxupquote{QThread}}

\sphinxAtStartPar
Worker thread for handling FLIM (Fluorescence Lifetime Measurement) processing in a separate thread 
to ensure that the GUI remains responsive during measurements.

\sphinxAtStartPar
This class performs the measurement tasks in the background, processes the data, and communicates with 
the main thread to update the GUI with the measurement status, results, and other relevant information.
\begin{description}
\sphinxlineitem{Signals:}\begin{itemize}
\item {} 
\sphinxAtStartPar
createdSignal: Signal emitted when the worker thread is created.

\item {} 
\sphinxAtStartPar
statusSignal: Signal emitted to update the status message in the GUI.

\item {} 
\sphinxAtStartPar
pointSignal: Signal emitted to update the current measurement point.

\item {} 
\sphinxAtStartPar
updateValues: Signal emitted to update measurement values and corresponding times.

\item {} 
\sphinxAtStartPar
updateLabel: Signal emitted to update a specific label in the GUI.

\item {} 
\sphinxAtStartPar
updateMeasurementsLabel: Signal emitted to update the label showing the number of measurements taken.

\end{itemize}

\end{description}
\begin{quote}\begin{description}
\sphinxlineitem{Parameters}\begin{itemize}
\item {} 
\sphinxAtStartPar
\sphinxstyleliteralstrong{\sphinxupquote{deviceStartChannel}} \textendash{} The start channel of the Tempico device used for measurement (TempicoChannel).

\item {} 
\sphinxAtStartPar
\sphinxstyleliteralstrong{\sphinxupquote{deviceStopChannel}} \textendash{} The stop channel of the Tempico device used for measurement (TempicoChannel).

\item {} 
\sphinxAtStartPar
\sphinxstyleliteralstrong{\sphinxupquote{binwidthText}} \textendash{} The selected bin width for the measurement (str).

\item {} 
\sphinxAtStartPar
\sphinxstyleliteralstrong{\sphinxupquote{numberMeasurements}} \textendash{} The number of measurements to be taken (int).

\item {} 
\sphinxAtStartPar
\sphinxstyleliteralstrong{\sphinxupquote{device}} \textendash{} The Tempico device used for performing measurements (Tempico).

\item {} 
\sphinxAtStartPar
\sphinxstyleliteralstrong{\sphinxupquote{TimeRange}} \textendash{} The time range for the measurements (int).

\end{itemize}

\sphinxlineitem{Returns}
\sphinxAtStartPar
None

\end{description}\end{quote}
\index{checkDeviceStatus() (FLIMGraphics.WorkerThreadFLIM method)@\spxentry{checkDeviceStatus()}\spxextra{FLIMGraphics.WorkerThreadFLIM method}}

\begin{fulllineitems}
\phantomsection\label{\detokenize{FLIMGraphics:FLIMGraphics.WorkerThreadFLIM.checkDeviceStatus}}
\pysigstartsignatures
\pysiglinewithargsret{\sphinxbfcode{\sphinxupquote{checkDeviceStatus}}}{}{}
\pysigstopsignatures
\sphinxAtStartPar
Checks the status of the device by attempting to read a parameter.

\sphinxAtStartPar
If an error occurs during the read operation, the measurement process is stopped.
\begin{quote}\begin{description}
\sphinxlineitem{Returns}
\sphinxAtStartPar
None

\end{description}\end{quote}

\end{fulllineitems}

\index{clear() (FLIMGraphics.WorkerThreadFLIM method)@\spxentry{clear()}\spxextra{FLIMGraphics.WorkerThreadFLIM method}}

\begin{fulllineitems}
\phantomsection\label{\detokenize{FLIMGraphics:FLIMGraphics.WorkerThreadFLIM.clear}}
\pysigstartsignatures
\pysiglinewithargsret{\sphinxbfcode{\sphinxupquote{clear}}}{}{}
\pysigstopsignatures
\sphinxAtStartPar
Clears the recorded start\sphinxhyphen{}stop differences and resets the total measurement count.

\sphinxAtStartPar
This function empties the list that stores the differences between start and stop measurements,
and sets the total number of measurements back to zero.
\begin{quote}\begin{description}
\sphinxlineitem{Returns}
\sphinxAtStartPar
None

\end{description}\end{quote}

\end{fulllineitems}

\index{createFLIMData() (FLIMGraphics.WorkerThreadFLIM method)@\spxentry{createFLIMData()}\spxextra{FLIMGraphics.WorkerThreadFLIM method}}

\begin{fulllineitems}
\phantomsection\label{\detokenize{FLIMGraphics:FLIMGraphics.WorkerThreadFLIM.createFLIMData}}
\pysigstartsignatures
\pysiglinewithargsret{\sphinxbfcode{\sphinxupquote{createFLIMData}}}{}{}
\pysigstopsignatures
\sphinxAtStartPar
Generates FLIM data from the start\sphinxhyphen{}stop differences and emits the calculated results to the main thread.

\sphinxAtStartPar
Processes the collected data to determine the number of counts within a given time based on the maximum value of start\sphinxhyphen{}stop differences.
Normalizes the data according to the determined units and calculates the histogram counts.
\begin{quote}\begin{description}
\sphinxlineitem{Returns}
\sphinxAtStartPar
None

\end{description}\end{quote}

\end{fulllineitems}

\index{createdSignal (FLIMGraphics.WorkerThreadFLIM attribute)@\spxentry{createdSignal}\spxextra{FLIMGraphics.WorkerThreadFLIM attribute}}

\begin{fulllineitems}
\phantomsection\label{\detokenize{FLIMGraphics:FLIMGraphics.WorkerThreadFLIM.createdSignal}}
\pysigstartsignatures
\pysigline{\sphinxbfcode{\sphinxupquote{createdSignal}}}
\pysigstopsignatures
\end{fulllineitems}

\index{getBinWidthNumber() (FLIMGraphics.WorkerThreadFLIM method)@\spxentry{getBinWidthNumber()}\spxextra{FLIMGraphics.WorkerThreadFLIM method}}

\begin{fulllineitems}
\phantomsection\label{\detokenize{FLIMGraphics:FLIMGraphics.WorkerThreadFLIM.getBinWidthNumber}}
\pysigstartsignatures
\pysiglinewithargsret{\sphinxbfcode{\sphinxupquote{getBinWidthNumber}}}{}{}
\pysigstopsignatures
\sphinxAtStartPar
Calculates the bin width in picoseconds based on user input in different units.

\sphinxAtStartPar
Extracts the numerical value and its corresponding unit from the user\sphinxhyphen{}provided text,
and converts it into picoseconds for further calculations.
\begin{quote}\begin{description}
\sphinxlineitem{Returns}
\sphinxAtStartPar
None

\end{description}\end{quote}

\end{fulllineitems}

\index{getUnits() (FLIMGraphics.WorkerThreadFLIM method)@\spxentry{getUnits()}\spxextra{FLIMGraphics.WorkerThreadFLIM method}}

\begin{fulllineitems}
\phantomsection\label{\detokenize{FLIMGraphics:FLIMGraphics.WorkerThreadFLIM.getUnits}}
\pysigstartsignatures
\pysiglinewithargsret{\sphinxbfcode{\sphinxupquote{getUnits}}}{\sphinxparam{\DUrole{n,n}{picosecondsValue}}}{}
\pysigstopsignatures
\sphinxAtStartPar
Determines the appropriate units for a given value in picoseconds.

\sphinxAtStartPar
Converts a given value in picoseconds to the most suitable time unit (picoseconds, nanoseconds, microseconds, or milliseconds).
\begin{quote}\begin{description}
\sphinxlineitem{Parameters}
\sphinxAtStartPar
\sphinxstyleliteralstrong{\sphinxupquote{picosecondsValue}} \textendash{} The value in picoseconds (float).

\sphinxlineitem{Returns}
\sphinxAtStartPar
A list containing the unit as a string and the factor by which to divide the value (list).

\end{description}\end{quote}

\end{fulllineitems}

\index{measurementMode() (FLIMGraphics.WorkerThreadFLIM method)@\spxentry{measurementMode()}\spxextra{FLIMGraphics.WorkerThreadFLIM method}}

\begin{fulllineitems}
\phantomsection\label{\detokenize{FLIMGraphics:FLIMGraphics.WorkerThreadFLIM.measurementMode}}
\pysigstartsignatures
\pysiglinewithargsret{\sphinxbfcode{\sphinxupquote{measurementMode}}}{}{}
\pysigstopsignatures
\sphinxAtStartPar
Sets the measurement mode for the device channels based on the specified time range.

\sphinxAtStartPar
If the time range is less than or equal to 500000 ps , sets all channels to mode 1.
Otherwise, sets all channels to mode 2.
\begin{quote}\begin{description}
\sphinxlineitem{Parameters}
\sphinxAtStartPar
\sphinxstyleliteralstrong{\sphinxupquote{self}} \textendash{} The instance of the class.

\sphinxlineitem{Returns}
\sphinxAtStartPar
None

\end{description}\end{quote}

\end{fulllineitems}

\index{pointSignal (FLIMGraphics.WorkerThreadFLIM attribute)@\spxentry{pointSignal}\spxextra{FLIMGraphics.WorkerThreadFLIM attribute}}

\begin{fulllineitems}
\phantomsection\label{\detokenize{FLIMGraphics:FLIMGraphics.WorkerThreadFLIM.pointSignal}}
\pysigstartsignatures
\pysigline{\sphinxbfcode{\sphinxupquote{pointSignal}}}
\pysigstopsignatures
\end{fulllineitems}

\index{run() (FLIMGraphics.WorkerThreadFLIM method)@\spxentry{run()}\spxextra{FLIMGraphics.WorkerThreadFLIM method}}

\begin{fulllineitems}
\phantomsection\label{\detokenize{FLIMGraphics:FLIMGraphics.WorkerThreadFLIM.run}}
\pysigstartsignatures
\pysiglinewithargsret{\sphinxbfcode{\sphinxupquote{run}}}{}{}
\pysigstopsignatures
\sphinxAtStartPar
Runs the measurement thread, continuously taking measurements based on user\sphinxhyphen{}selected values
and passing them to the graphs.
\begin{quote}\begin{description}
\sphinxlineitem{Returns}
\sphinxAtStartPar
None

\end{description}\end{quote}

\end{fulllineitems}

\index{staticMetaObject (FLIMGraphics.WorkerThreadFLIM attribute)@\spxentry{staticMetaObject}\spxextra{FLIMGraphics.WorkerThreadFLIM attribute}}

\begin{fulllineitems}
\phantomsection\label{\detokenize{FLIMGraphics:FLIMGraphics.WorkerThreadFLIM.staticMetaObject}}
\pysigstartsignatures
\pysigline{\sphinxbfcode{\sphinxupquote{staticMetaObject}}\sphinxbfcode{\sphinxupquote{\DUrole{w,w}{  }\DUrole{p,p}{=}\DUrole{w,w}{  }\textless{}PySide2.QtCore.QMetaObject object\textgreater{}}}}
\pysigstopsignatures
\end{fulllineitems}

\index{statusSignal (FLIMGraphics.WorkerThreadFLIM attribute)@\spxentry{statusSignal}\spxextra{FLIMGraphics.WorkerThreadFLIM attribute}}

\begin{fulllineitems}
\phantomsection\label{\detokenize{FLIMGraphics:FLIMGraphics.WorkerThreadFLIM.statusSignal}}
\pysigstartsignatures
\pysigline{\sphinxbfcode{\sphinxupquote{statusSignal}}}
\pysigstopsignatures
\end{fulllineitems}

\index{stop() (FLIMGraphics.WorkerThreadFLIM method)@\spxentry{stop()}\spxextra{FLIMGraphics.WorkerThreadFLIM method}}

\begin{fulllineitems}
\phantomsection\label{\detokenize{FLIMGraphics:FLIMGraphics.WorkerThreadFLIM.stop}}
\pysigstartsignatures
\pysiglinewithargsret{\sphinxbfcode{\sphinxupquote{stop}}}{}{}
\pysigstopsignatures
\sphinxAtStartPar
Stops the measurement process by setting the running flag to False.

\sphinxAtStartPar
Emits a status message indicating that the measurement is ending and updates the status bar color to yellow.
\begin{quote}\begin{description}
\sphinxlineitem{Returns}
\sphinxAtStartPar
None

\end{description}\end{quote}

\end{fulllineitems}

\index{takeMeasurements() (FLIMGraphics.WorkerThreadFLIM method)@\spxentry{takeMeasurements()}\spxextra{FLIMGraphics.WorkerThreadFLIM method}}

\begin{fulllineitems}
\phantomsection\label{\detokenize{FLIMGraphics:FLIMGraphics.WorkerThreadFLIM.takeMeasurements}}
\pysigstartsignatures
\pysiglinewithargsret{\sphinxbfcode{\sphinxupquote{takeMeasurements}}}{\sphinxparam{\DUrole{n,n}{percentage}}}{}
\pysigstopsignatures
\sphinxAtStartPar
Takes measurements from the device and updates the start\sphinxhyphen{}stop differences list.

\sphinxAtStartPar
Configures the device for measurement runs and checks the input and stop channels for data.
If no measurements are detected, emits a status signal to update the main thread about the current measurement state.
\begin{quote}\begin{description}
\sphinxlineitem{Parameters}
\sphinxAtStartPar
\sphinxstyleliteralstrong{\sphinxupquote{percentage}} \textendash{} The percentage of measurement completion (float).

\sphinxlineitem{Returns}
\sphinxAtStartPar
None

\end{description}\end{quote}

\end{fulllineitems}

\index{updateLabel (FLIMGraphics.WorkerThreadFLIM attribute)@\spxentry{updateLabel}\spxextra{FLIMGraphics.WorkerThreadFLIM attribute}}

\begin{fulllineitems}
\phantomsection\label{\detokenize{FLIMGraphics:FLIMGraphics.WorkerThreadFLIM.updateLabel}}
\pysigstartsignatures
\pysigline{\sphinxbfcode{\sphinxupquote{updateLabel}}}
\pysigstopsignatures
\end{fulllineitems}

\index{updateMeasurementsLabel (FLIMGraphics.WorkerThreadFLIM attribute)@\spxentry{updateMeasurementsLabel}\spxextra{FLIMGraphics.WorkerThreadFLIM attribute}}

\begin{fulllineitems}
\phantomsection\label{\detokenize{FLIMGraphics:FLIMGraphics.WorkerThreadFLIM.updateMeasurementsLabel}}
\pysigstartsignatures
\pysigline{\sphinxbfcode{\sphinxupquote{updateMeasurementsLabel}}}
\pysigstopsignatures
\end{fulllineitems}

\index{updateValues (FLIMGraphics.WorkerThreadFLIM attribute)@\spxentry{updateValues}\spxextra{FLIMGraphics.WorkerThreadFLIM attribute}}

\begin{fulllineitems}
\phantomsection\label{\detokenize{FLIMGraphics:FLIMGraphics.WorkerThreadFLIM.updateValues}}
\pysigstartsignatures
\pysigline{\sphinxbfcode{\sphinxupquote{updateValues}}}
\pysigstopsignatures
\end{fulllineitems}


\end{fulllineitems}


\sphinxstepscope


\chapter{StartStopHist module}
\label{\detokenize{StartStopHist:module-StartStopHist}}\label{\detokenize{StartStopHist:startstophist-module}}\label{\detokenize{StartStopHist::doc}}\index{module@\spxentry{module}!StartStopHist@\spxentry{StartStopHist}}\index{StartStopHist@\spxentry{StartStopHist}!module@\spxentry{module}}\index{StartStopLogic (class in StartStopHist)@\spxentry{StartStopLogic}\spxextra{class in StartStopHist}}

\begin{fulllineitems}
\phantomsection\label{\detokenize{StartStopHist:StartStopHist.StartStopLogic}}
\pysigstartsignatures
\pysiglinewithargsret{\sphinxbfcode{\sphinxupquote{class\DUrole{w,w}{  }}}\sphinxcode{\sphinxupquote{StartStopHist.}}\sphinxbfcode{\sphinxupquote{StartStopLogic}}}{\sphinxparam{\DUrole{n,n}{parent}}\sphinxparamcomma \sphinxparam{\DUrole{n,n}{disconnect}}\sphinxparamcomma \sphinxparam{\DUrole{n,n}{device}}\sphinxparamcomma \sphinxparam{\DUrole{n,n}{check1}}\sphinxparamcomma \sphinxparam{\DUrole{n,n}{check2}}\sphinxparamcomma \sphinxparam{\DUrole{n,n}{check3}}\sphinxparamcomma \sphinxparam{\DUrole{n,n}{check4}}\sphinxparamcomma \sphinxparam{\DUrole{n,n}{startbutton}}\sphinxparamcomma \sphinxparam{\DUrole{n,n}{stopbutton}}\sphinxparamcomma \sphinxparam{\DUrole{n,n}{savebutton}}\sphinxparamcomma \sphinxparam{\DUrole{n,n}{save\_graph\_1}}\sphinxparamcomma \sphinxparam{\DUrole{n,n}{clear\_channel\_A}}\sphinxparamcomma \sphinxparam{\DUrole{n,n}{clear\_channel\_B}}\sphinxparamcomma \sphinxparam{\DUrole{n,n}{clear\_channel\_C}}\sphinxparamcomma \sphinxparam{\DUrole{n,n}{clear\_channel\_D}}\sphinxparamcomma \sphinxparam{\DUrole{n,n}{connect}}\sphinxparamcomma \sphinxparam{\DUrole{n,n}{mainWindow}}\sphinxparamcomma \sphinxparam{\DUrole{n,n}{statusValue}}\sphinxparamcomma \sphinxparam{\DUrole{n,n}{statusPoint}}\sphinxparamcomma \sphinxparam{\DUrole{o,o}{*}\DUrole{n,n}{args}}\sphinxparamcomma \sphinxparam{\DUrole{o,o}{**}\DUrole{n,n}{kwargs}}}{}
\pysigstopsignatures
\sphinxAtStartPar
Bases: \sphinxcode{\sphinxupquote{object}}

\sphinxAtStartPar
This class handles the functionality for managing buttons, checkboxes, and graphs for histograms related to the Start\sphinxhyphen{}Stop measurements. 
It controls which channels are active, manages the display of histograms, and records the data from the Tempico device based on the user’s interactions. 
It also handles the zooming feature for creating bars according to the time range received from the device.
\begin{quote}\begin{description}
\sphinxlineitem{Parameters}\begin{itemize}
\item {} 
\sphinxAtStartPar
\sphinxstyleliteralstrong{\sphinxupquote{parent}} \textendash{} The parent QWindow for the logic.

\item {} 
\sphinxAtStartPar
\sphinxstyleliteralstrong{\sphinxupquote{disconnect}} \textendash{} The QPushButton used to disconnect.

\item {} 
\sphinxAtStartPar
\sphinxstyleliteralstrong{\sphinxupquote{device}} \textendash{} The Tempico device class that handles the measurements.

\item {} 
\sphinxAtStartPar
\sphinxstyleliteralstrong{\sphinxupquote{check4}} (\sphinxstyleliteralemphasis{\sphinxupquote{check1}}\sphinxstyleliteralemphasis{\sphinxupquote{, }}\sphinxstyleliteralemphasis{\sphinxupquote{check2}}\sphinxstyleliteralemphasis{\sphinxupquote{, }}\sphinxstyleliteralemphasis{\sphinxupquote{check3}}\sphinxstyleliteralemphasis{\sphinxupquote{,}}) \textendash{} QCheckBoxes that control the visibility of each histogram.

\item {} 
\sphinxAtStartPar
\sphinxstyleliteralstrong{\sphinxupquote{stopbutton}} (\sphinxstyleliteralemphasis{\sphinxupquote{startbutton}}\sphinxstyleliteralemphasis{\sphinxupquote{,}}) \textendash{} QPushButtons to start and stop the measurement process.

\item {} 
\sphinxAtStartPar
\sphinxstyleliteralstrong{\sphinxupquote{save\_graph\_1}} (\sphinxstyleliteralemphasis{\sphinxupquote{savebutton}}\sphinxstyleliteralemphasis{\sphinxupquote{,}}) \textendash{} QPushButtons to save data and save the graph.

\item {} 
\sphinxAtStartPar
\sphinxstyleliteralstrong{\sphinxupquote{clear\_channel\_D}} (\sphinxstyleliteralemphasis{\sphinxupquote{clear\_channel\_A}}\sphinxstyleliteralemphasis{\sphinxupquote{, }}\sphinxstyleliteralemphasis{\sphinxupquote{clear\_channel\_B}}\sphinxstyleliteralemphasis{\sphinxupquote{, }}\sphinxstyleliteralemphasis{\sphinxupquote{clear\_channel\_C}}\sphinxstyleliteralemphasis{\sphinxupquote{,}}) \textendash{} QPushButtons to clear data for each respective channel.

\item {} 
\sphinxAtStartPar
\sphinxstyleliteralstrong{\sphinxupquote{connect}} \textendash{} QPushButton to connect to the Tempico device.

\item {} 
\sphinxAtStartPar
\sphinxstyleliteralstrong{\sphinxupquote{mainWindow}} \textendash{} The main window (QWindow) for the GUI.

\item {} 
\sphinxAtStartPar
\sphinxstyleliteralstrong{\sphinxupquote{statusPoint}} (\sphinxstyleliteralemphasis{\sphinxupquote{statusValue}}\sphinxstyleliteralemphasis{\sphinxupquote{,}}) \textendash{} QLabel widgets for displaying status information (e.g., values and points).

\end{itemize}

\end{description}\end{quote}
\index{autoRangeA() (StartStopHist.StartStopLogic method)@\spxentry{autoRangeA()}\spxextra{StartStopHist.StartStopLogic method}}

\begin{fulllineitems}
\phantomsection\label{\detokenize{StartStopHist:StartStopHist.StartStopLogic.autoRangeA}}
\pysigstartsignatures
\pysiglinewithargsret{\sphinxbfcode{\sphinxupquote{autoRangeA}}}{}{}
\pysigstopsignatures
\sphinxAtStartPar
Automatically adjusts the zoom level of channel A to the maximum value in the data.

\sphinxAtStartPar
The function sets a sentinel for zoom changes, checks if there are any data points in 
channel A, and if so, finds the maximum value. It then calls the changeZoomMax function
to adjust the zoom accordingly.
\begin{quote}\begin{description}
\sphinxlineitem{Returns}
\sphinxAtStartPar
None

\end{description}\end{quote}

\end{fulllineitems}

\index{autoRangeB() (StartStopHist.StartStopLogic method)@\spxentry{autoRangeB()}\spxextra{StartStopHist.StartStopLogic method}}

\begin{fulllineitems}
\phantomsection\label{\detokenize{StartStopHist:StartStopHist.StartStopLogic.autoRangeB}}
\pysigstartsignatures
\pysiglinewithargsret{\sphinxbfcode{\sphinxupquote{autoRangeB}}}{}{}
\pysigstopsignatures
\sphinxAtStartPar
Automatically adjusts the zoom level of channel B to the maximum value in the data.

\sphinxAtStartPar
The function sets a sentinel for zoom changes, checks if there are any data points in 
channel B, and if so, finds the maximum value. It then calls the changeZoomMax function
to adjust the zoom accordingly.
\begin{quote}\begin{description}
\sphinxlineitem{Returns}
\sphinxAtStartPar
None

\end{description}\end{quote}

\end{fulllineitems}

\index{autoRangeC() (StartStopHist.StartStopLogic method)@\spxentry{autoRangeC()}\spxextra{StartStopHist.StartStopLogic method}}

\begin{fulllineitems}
\phantomsection\label{\detokenize{StartStopHist:StartStopHist.StartStopLogic.autoRangeC}}
\pysigstartsignatures
\pysiglinewithargsret{\sphinxbfcode{\sphinxupquote{autoRangeC}}}{}{}
\pysigstopsignatures
\sphinxAtStartPar
Automatically adjusts the zoom level of channel C to the maximum value in the data.

\sphinxAtStartPar
The function sets a sentinel for zoom changes, checks if there are any data points in 
channel C, and if so, finds the maximum value. It then calls the changeZoomMax function
to adjust the zoom accordingly.
\begin{quote}\begin{description}
\sphinxlineitem{Returns}
\sphinxAtStartPar
None

\end{description}\end{quote}

\end{fulllineitems}

\index{autoRangeD() (StartStopHist.StartStopLogic method)@\spxentry{autoRangeD()}\spxextra{StartStopHist.StartStopLogic method}}

\begin{fulllineitems}
\phantomsection\label{\detokenize{StartStopHist:StartStopHist.StartStopLogic.autoRangeD}}
\pysigstartsignatures
\pysiglinewithargsret{\sphinxbfcode{\sphinxupquote{autoRangeD}}}{}{}
\pysigstopsignatures
\sphinxAtStartPar
Automatically adjusts the zoom level of channel D to the maximum value in the data.

\sphinxAtStartPar
The function sets a sentinel for zoom changes, checks if there are any data points in 
channel D, and if so, finds the maximum value. It then calls the changeZoomMax function
to adjust the zoom accordingly.
\begin{quote}\begin{description}
\sphinxlineitem{Returns}
\sphinxAtStartPar
None

\end{description}\end{quote}

\end{fulllineitems}

\index{changeColorThread() (StartStopHist.StartStopLogic method)@\spxentry{changeColorThread()}\spxextra{StartStopHist.StartStopLogic method}}

\begin{fulllineitems}
\phantomsection\label{\detokenize{StartStopHist:StartStopHist.StartStopLogic.changeColorThread}}
\pysigstartsignatures
\pysiglinewithargsret{\sphinxbfcode{\sphinxupquote{changeColorThread}}}{\sphinxparam{\DUrole{n,n}{color}}}{}
\pysigstopsignatures
\end{fulllineitems}

\index{changeStatusColor() (StartStopHist.StartStopLogic method)@\spxentry{changeStatusColor()}\spxextra{StartStopHist.StartStopLogic method}}

\begin{fulllineitems}
\phantomsection\label{\detokenize{StartStopHist:StartStopHist.StartStopLogic.changeStatusColor}}
\pysigstartsignatures
\pysiglinewithargsret{\sphinxbfcode{\sphinxupquote{changeStatusColor}}}{\sphinxparam{\DUrole{n,n}{color}}}{}
\pysigstopsignatures
\sphinxAtStartPar
Changes the color of the status point in the status bar based on the input value.

\sphinxAtStartPar
This method updates the status point’s color by drawing a filled circle with the specified color.
The mapping of numerical values to colors is as follows:
\sphinxhyphen{} 0: Gray
\sphinxhyphen{} 1: Green
\sphinxhyphen{} 2: Yellow
\sphinxhyphen{} 3: Orange
\begin{quote}\begin{description}
\sphinxlineitem{Parameters}
\sphinxAtStartPar
\sphinxstyleliteralstrong{\sphinxupquote{color}} \textendash{} A numerical value representing the desired color (int).
0 for gray, 1 for green, 2 for yellow, 3 for orange.

\sphinxlineitem{Returns}
\sphinxAtStartPar
None

\end{description}\end{quote}

\end{fulllineitems}

\index{changeStatusThread() (StartStopHist.StartStopLogic method)@\spxentry{changeStatusThread()}\spxextra{StartStopHist.StartStopLogic method}}

\begin{fulllineitems}
\phantomsection\label{\detokenize{StartStopHist:StartStopHist.StartStopLogic.changeStatusThread}}
\pysigstartsignatures
\pysiglinewithargsret{\sphinxbfcode{\sphinxupquote{changeStatusThread}}}{\sphinxparam{\DUrole{n,n}{newText}}}{}
\pysigstopsignatures
\sphinxAtStartPar
Updates the text displayed in the status bar.

\sphinxAtStartPar
This method sets the status value text to the specified new text, 
reflecting the current status or information relevant to the user.
\begin{quote}\begin{description}
\sphinxlineitem{Parameters}
\sphinxAtStartPar
\sphinxstyleliteralstrong{\sphinxupquote{newText}} \textendash{} The text to display in the status bar (str).

\sphinxlineitem{Returns}
\sphinxAtStartPar
None

\end{description}\end{quote}

\end{fulllineitems}

\index{changeZoomMax() (StartStopHist.StartStopLogic method)@\spxentry{changeZoomMax()}\spxextra{StartStopHist.StartStopLogic method}}

\begin{fulllineitems}
\phantomsection\label{\detokenize{StartStopHist:StartStopHist.StartStopLogic.changeZoomMax}}
\pysigstartsignatures
\pysiglinewithargsret{\sphinxbfcode{\sphinxupquote{changeZoomMax}}}{\sphinxparam{\DUrole{n,n}{newMaxValue}}\sphinxparamcomma \sphinxparam{\DUrole{n,n}{channel}}}{}
\pysigstopsignatures
\sphinxAtStartPar
Automatically adjusts the zoom level of the plot to the maximum value found.

\sphinxAtStartPar
The function sets the x\sphinxhyphen{}axis range of the specified channel’s plot to {[}0, newMaxValue{]}.
It considers whether measurements have been taken and limits the zoom change based on 
the sentinel zoom change counters.
\begin{quote}\begin{description}
\sphinxlineitem{Parameters}\begin{itemize}
\item {} 
\sphinxAtStartPar
\sphinxstyleliteralstrong{\sphinxupquote{newMaxValue}} \textendash{} The new maximum value for the zoom range (float).

\item {} 
\sphinxAtStartPar
\sphinxstyleliteralstrong{\sphinxupquote{channel}} \textendash{} The channel to which the zoom change applies (str).
Expected values are ‘A’, ‘B’, ‘C’, or ‘D’.

\end{itemize}

\sphinxlineitem{Returns}
\sphinxAtStartPar
None

\end{description}\end{quote}

\end{fulllineitems}

\index{clear\_a() (StartStopHist.StartStopLogic method)@\spxentry{clear\_a()}\spxextra{StartStopHist.StartStopLogic method}}

\begin{fulllineitems}
\phantomsection\label{\detokenize{StartStopHist:StartStopHist.StartStopLogic.clear_a}}
\pysigstartsignatures
\pysiglinewithargsret{\sphinxbfcode{\sphinxupquote{clear\_a}}}{}{}
\pysigstopsignatures
\sphinxAtStartPar
Clears the lists containing measured data for channel A by resetting them to empty lists.
\begin{quote}\begin{description}
\sphinxlineitem{Returns}
\sphinxAtStartPar
None

\end{description}\end{quote}

\end{fulllineitems}

\index{clear\_b() (StartStopHist.StartStopLogic method)@\spxentry{clear\_b()}\spxextra{StartStopHist.StartStopLogic method}}

\begin{fulllineitems}
\phantomsection\label{\detokenize{StartStopHist:StartStopHist.StartStopLogic.clear_b}}
\pysigstartsignatures
\pysiglinewithargsret{\sphinxbfcode{\sphinxupquote{clear\_b}}}{}{}
\pysigstopsignatures
\sphinxAtStartPar
Clears the lists containing measured data for channel B by resetting them to empty lists.
\begin{quote}\begin{description}
\sphinxlineitem{Returns}
\sphinxAtStartPar
None

\end{description}\end{quote}

\end{fulllineitems}

\index{clear\_c() (StartStopHist.StartStopLogic method)@\spxentry{clear\_c()}\spxextra{StartStopHist.StartStopLogic method}}

\begin{fulllineitems}
\phantomsection\label{\detokenize{StartStopHist:StartStopHist.StartStopLogic.clear_c}}
\pysigstartsignatures
\pysiglinewithargsret{\sphinxbfcode{\sphinxupquote{clear\_c}}}{}{}
\pysigstopsignatures
\sphinxAtStartPar
Clears the lists containing measured data for channel C by resetting them to empty lists.
\begin{quote}\begin{description}
\sphinxlineitem{Returns}
\sphinxAtStartPar
None

\end{description}\end{quote}

\end{fulllineitems}

\index{clear\_d() (StartStopHist.StartStopLogic method)@\spxentry{clear\_d()}\spxextra{StartStopHist.StartStopLogic method}}

\begin{fulllineitems}
\phantomsection\label{\detokenize{StartStopHist:StartStopHist.StartStopLogic.clear_d}}
\pysigstartsignatures
\pysiglinewithargsret{\sphinxbfcode{\sphinxupquote{clear\_d}}}{}{}
\pysigstopsignatures
\sphinxAtStartPar
Clears the lists containing measured data for channel D by resetting them to empty lists.
\begin{quote}\begin{description}
\sphinxlineitem{Returns}
\sphinxAtStartPar
None

\end{description}\end{quote}

\end{fulllineitems}

\index{createDialog() (StartStopHist.StartStopLogic method)@\spxentry{createDialog()}\spxextra{StartStopHist.StartStopLogic method}}

\begin{fulllineitems}
\phantomsection\label{\detokenize{StartStopHist:StartStopHist.StartStopLogic.createDialog}}
\pysigstartsignatures
\pysiglinewithargsret{\sphinxbfcode{\sphinxupquote{createDialog}}}{}{}
\pysigstopsignatures
\sphinxAtStartPar
Creates a dialog box to notify the user when the connection with the device has been lost.

\sphinxAtStartPar
This method displays a critical error message box indicating that the connection with the device
has been lost. It also stops any ongoing graphic updates, hides relevant graphics, 
and disables the disconnect button while enabling the connect button.
\begin{quote}\begin{description}
\sphinxlineitem{Returns}
\sphinxAtStartPar
None

\end{description}\end{quote}

\end{fulllineitems}

\index{create\_graphs() (StartStopHist.StartStopLogic method)@\spxentry{create\_graphs()}\spxextra{StartStopHist.StartStopLogic method}}

\begin{fulllineitems}
\phantomsection\label{\detokenize{StartStopHist:StartStopHist.StartStopLogic.create_graphs}}
\pysigstartsignatures
\pysiglinewithargsret{\sphinxbfcode{\sphinxupquote{create\_graphs}}}{}{}
\pysigstopsignatures
\sphinxAtStartPar
Creates histograms and plots for the selected channels (A, B, C, D) and arranges them on the GUI.

\sphinxAtStartPar
This function initializes the graphical components and assigns the appropriate properties to the plots based on the selected channels.
It also manages the layout of the graphs, resizing them according to the number of selected channels (1, 2, 3, or 4). 
After setting up the graphical components, the function starts a worker thread that handles the measurement process in parallel with the GUI thread, ensuring the UI remains responsive.
\begin{quote}\begin{description}
\sphinxlineitem{Returns}
\sphinxAtStartPar
None

\end{description}\end{quote}

\end{fulllineitems}

\index{dialogChangeMode() (StartStopHist.StartStopLogic method)@\spxentry{dialogChangeMode()}\spxextra{StartStopHist.StartStopLogic method}}

\begin{fulllineitems}
\phantomsection\label{\detokenize{StartStopHist:StartStopHist.StartStopLogic.dialogChangeMode}}
\pysigstartsignatures
\pysiglinewithargsret{\sphinxbfcode{\sphinxupquote{dialogChangeMode}}}{\sphinxparam{\DUrole{n,n}{channel}}}{}
\pysigstopsignatures
\sphinxAtStartPar
Opens a dialog box to ask the user if they want to change the mode for the selected channel.

\sphinxAtStartPar
The dialog informs the user that the collected data for the specified channel falls 
mostly outside the reliable range of mode 1 and prompts them to switch to mode 2. 
If the user confirms, the function updates the mode and processes the data accordingly.
\begin{quote}\begin{description}
\sphinxlineitem{Parameters}
\sphinxAtStartPar
\sphinxstyleliteralstrong{\sphinxupquote{channel}} \textendash{} The channel for which to change the mode (str). 
Expected values are ‘channel A’, ‘channel B’, ‘channel C’, or ‘channel D’.

\sphinxlineitem{Returns}
\sphinxAtStartPar
None

\end{description}\end{quote}

\end{fulllineitems}

\index{hide\_graphic2() (StartStopHist.StartStopLogic method)@\spxentry{hide\_graphic2()}\spxextra{StartStopHist.StartStopLogic method}}

\begin{fulllineitems}
\phantomsection\label{\detokenize{StartStopHist:StartStopHist.StartStopLogic.hide_graphic2}}
\pysigstartsignatures
\pysiglinewithargsret{\sphinxbfcode{\sphinxupquote{hide\_graphic2}}}{}{}
\pysigstopsignatures
\sphinxAtStartPar
Disables the start and stop buttons in the graphical interface.

\sphinxAtStartPar
This function is typically used to prevent user interaction during certain operations.
\begin{quote}\begin{description}
\sphinxlineitem{Returns}
\sphinxAtStartPar
None

\end{description}\end{quote}

\end{fulllineitems}

\index{save\_graphic() (StartStopHist.StartStopLogic method)@\spxentry{save\_graphic()}\spxextra{StartStopHist.StartStopLogic method}}

\begin{fulllineitems}
\phantomsection\label{\detokenize{StartStopHist:StartStopHist.StartStopLogic.save_graphic}}
\pysigstartsignatures
\pysiglinewithargsret{\sphinxbfcode{\sphinxupquote{save\_graphic}}}{}{}
\pysigstopsignatures
\sphinxAtStartPar
Saves the current data according to the selected format specified in the dialog box.

\sphinxAtStartPar
The function starts by retrieving the default histogram name and the current 
date, which is formatted to create a unique filename. It then initializes lists 
to hold filenames, data, settings, and column names for the saved files.

\sphinxAtStartPar
A dialog box is displayed for the user to select the desired file format 
(txt, csv, or dat) for saving the data. Once the user accepts the selection, 
the function checks if the selected format has already been saved before.

\sphinxAtStartPar
If the selected format has not been saved, it collects data from the various 
channels (A, B, C, D) if their corresponding sentinel variables indicate 
they should be saved. It retrieves the average cycles, mode, number of stops, 
stop edge, and stop mask for each channel to include in the settings. The 
function then attempts to save the collected data in the specified format.

\sphinxAtStartPar
If the save operation is successful, a message box is displayed, confirming 
the successful save and showing the folder path and file names. If an error 
occurs during the save process, an error message box is shown.

\sphinxAtStartPar
If the selected format has already been saved, a message box is displayed 
with the previous save information.
\begin{quote}\begin{description}
\sphinxlineitem{Returns}
\sphinxAtStartPar
None

\end{description}\end{quote}

\end{fulllineitems}

\index{save\_plots() (StartStopHist.StartStopLogic method)@\spxentry{save\_plots()}\spxextra{StartStopHist.StartStopLogic method}}

\begin{fulllineitems}
\phantomsection\label{\detokenize{StartStopHist:StartStopHist.StartStopLogic.save_plots}}
\pysigstartsignatures
\pysiglinewithargsret{\sphinxbfcode{\sphinxupquote{save\_plots}}}{}{}
\pysigstopsignatures
\sphinxAtStartPar
Saves the current plots in the selected image format.

\sphinxAtStartPar
This method opens a dialog for the user to select an image format (PNG, TIFF, or JPG) 
and saves the plots for channels A, B, C, and D if their respective flags are set to True. 
The plots are saved with a timestamp in the specified format in the default folder path.
\begin{quote}\begin{description}
\sphinxlineitem{Returns}
\sphinxAtStartPar
None

\end{description}\end{quote}

\end{fulllineitems}

\index{show\_graphic() (StartStopHist.StartStopLogic method)@\spxentry{show\_graphic()}\spxextra{StartStopHist.StartStopLogic method}}

\begin{fulllineitems}
\phantomsection\label{\detokenize{StartStopHist:StartStopHist.StartStopLogic.show_graphic}}
\pysigstartsignatures
\pysiglinewithargsret{\sphinxbfcode{\sphinxupquote{show\_graphic}}}{\sphinxparam{\DUrole{n,n}{device\_new}}}{}
\pysigstopsignatures
\sphinxAtStartPar
Sets the device and its channels, enabling the start button for measurement.
\begin{quote}\begin{description}
\sphinxlineitem{Parameters}
\sphinxAtStartPar
\sphinxstyleliteralstrong{\sphinxupquote{device\_new}} \textendash{} The new device to be set (DeviceType).

\sphinxlineitem{Returns}
\sphinxAtStartPar
None

\end{description}\end{quote}

\end{fulllineitems}

\index{start\_graphic() (StartStopHist.StartStopLogic method)@\spxentry{start\_graphic()}\spxextra{StartStopHist.StartStopLogic method}}

\begin{fulllineitems}
\phantomsection\label{\detokenize{StartStopHist:StartStopHist.StartStopLogic.start_graphic}}
\pysigstartsignatures
\pysiglinewithargsret{\sphinxbfcode{\sphinxupquote{start\_graphic}}}{}{}
\pysigstopsignatures
\sphinxAtStartPar
Starts the graphical representation of the measurement based on the selected channels.

\sphinxAtStartPar
If no channels are selected, it prompts the user to select at least one channel before proceeding. 
It also disables certain buttons to prevent interaction during the measurement process.
\begin{quote}\begin{description}
\sphinxlineitem{Returns}
\sphinxAtStartPar
None

\end{description}\end{quote}

\end{fulllineitems}

\index{stop\_graphic() (StartStopHist.StartStopLogic method)@\spxentry{stop\_graphic()}\spxextra{StartStopHist.StartStopLogic method}}

\begin{fulllineitems}
\phantomsection\label{\detokenize{StartStopHist:StartStopHist.StartStopLogic.stop_graphic}}
\pysigstartsignatures
\pysiglinewithargsret{\sphinxbfcode{\sphinxupquote{stop\_graphic}}}{}{}
\pysigstopsignatures
\sphinxAtStartPar
Enables the functions to initiate a new measurement while disabling the buttons to clear graphs. 
It also updates the status bar text and color.
\begin{quote}\begin{description}
\sphinxlineitem{Returns}
\sphinxAtStartPar
None

\end{description}\end{quote}

\end{fulllineitems}

\index{threadComplete() (StartStopHist.StartStopLogic method)@\spxentry{threadComplete()}\spxextra{StartStopHist.StartStopLogic method}}

\begin{fulllineitems}
\phantomsection\label{\detokenize{StartStopHist:StartStopHist.StartStopLogic.threadComplete}}
\pysigstartsignatures
\pysiglinewithargsret{\sphinxbfcode{\sphinxupquote{threadComplete}}}{}{}
\pysigstopsignatures
\sphinxAtStartPar
Handles the completion of the measurement thread by enabling the ability to switch between windows. 
It also calls the stop\_graphic() function to update the UI accordingly.
\begin{quote}\begin{description}
\sphinxlineitem{Returns}
\sphinxAtStartPar
None

\end{description}\end{quote}

\end{fulllineitems}

\index{threadRunning() (StartStopHist.StartStopLogic method)@\spxentry{threadRunning()}\spxextra{StartStopHist.StartStopLogic method}}

\begin{fulllineitems}
\phantomsection\label{\detokenize{StartStopHist:StartStopHist.StartStopLogic.threadRunning}}
\pysigstartsignatures
\pysiglinewithargsret{\sphinxbfcode{\sphinxupquote{threadRunning}}}{\sphinxparam{\DUrole{n,n}{status}}}{}
\pysigstopsignatures
\sphinxAtStartPar
Updates the thread creation status.

\sphinxAtStartPar
This method sets the threadCreatedSentinel attribute to indicate whether a thread has been created 
based on the provided status value.
\begin{quote}\begin{description}
\sphinxlineitem{Parameters}
\sphinxAtStartPar
\sphinxstyleliteralstrong{\sphinxupquote{status}} \textendash{} The status indicating whether a thread has been created (int). 
Use 0 to indicate that the thread has been created, and 1 to indicate that it has not.

\sphinxlineitem{Returns}
\sphinxAtStartPar
None

\end{description}\end{quote}

\end{fulllineitems}

\index{updateDataPure() (StartStopHist.StartStopLogic method)@\spxentry{updateDataPure()}\spxextra{StartStopHist.StartStopLogic method}}

\begin{fulllineitems}
\phantomsection\label{\detokenize{StartStopHist:StartStopHist.StartStopLogic.updateDataPure}}
\pysigstartsignatures
\pysiglinewithargsret{\sphinxbfcode{\sphinxupquote{updateDataPure}}}{\sphinxparam{\DUrole{n,n}{value}}\sphinxparamcomma \sphinxparam{\DUrole{n,n}{channel}}}{}
\pysigstopsignatures
\sphinxAtStartPar
Updates the raw data values for the specified channel.

\sphinxAtStartPar
This method appends the new raw value (in picoseconds) to the list of data for the specified 
channel (A, B, C, or D).
\begin{quote}\begin{description}
\sphinxlineitem{Parameters}\begin{itemize}
\item {} 
\sphinxAtStartPar
\sphinxstyleliteralstrong{\sphinxupquote{value}} \textendash{} The raw data value to append to the data list (float or int). 
This value is in picoseconds.

\item {} 
\sphinxAtStartPar
\sphinxstyleliteralstrong{\sphinxupquote{channel}} \textendash{} The channel identifier for which the value is to be updated (str). 
Accepted values are “A”, “B”, “C”, and “D”.

\end{itemize}

\sphinxlineitem{Returns}
\sphinxAtStartPar
None

\end{description}\end{quote}

\end{fulllineitems}

\index{updateSignal() (StartStopHist.StartStopLogic method)@\spxentry{updateSignal()}\spxextra{StartStopHist.StartStopLogic method}}

\begin{fulllineitems}
\phantomsection\label{\detokenize{StartStopHist:StartStopHist.StartStopLogic.updateSignal}}
\pysigstartsignatures
\pysiglinewithargsret{\sphinxbfcode{\sphinxupquote{updateSignal}}}{\sphinxparam{\DUrole{n,n}{value}}\sphinxparamcomma \sphinxparam{\DUrole{n,n}{channel}}}{}
\pysigstopsignatures
\sphinxAtStartPar
Updates the normalized data values for the specified channel and refreshes the corresponding histogram.

\sphinxAtStartPar
This method appends the new value to the list of normalized data for the specified channel 
(A, B, C, or D) and calls the update\_histogram method to refresh the histogram with the 
updated data.
\begin{quote}\begin{description}
\sphinxlineitem{Parameters}\begin{itemize}
\item {} 
\sphinxAtStartPar
\sphinxstyleliteralstrong{\sphinxupquote{value}} \textendash{} The normalized value to append to the data list (float or int).

\item {} 
\sphinxAtStartPar
\sphinxstyleliteralstrong{\sphinxupquote{channel}} \textendash{} The channel identifier for which the value is to be updated (str). 
Accepted values are “A”, “B”, “C”, and “D”.

\end{itemize}

\sphinxlineitem{Returns}
\sphinxAtStartPar
None

\end{description}\end{quote}

\end{fulllineitems}

\index{update\_histogram() (StartStopHist.StartStopLogic method)@\spxentry{update\_histogram()}\spxextra{StartStopHist.StartStopLogic method}}

\begin{fulllineitems}
\phantomsection\label{\detokenize{StartStopHist:StartStopHist.StartStopLogic.update_histogram}}
\pysigstartsignatures
\pysiglinewithargsret{\sphinxbfcode{\sphinxupquote{update\_histogram}}}{\sphinxparam{\DUrole{n,n}{data}}\sphinxparamcomma \sphinxparam{\DUrole{n,n}{curve}}\sphinxparamcomma \sphinxparam{\DUrole{n,n}{indexChannel}}}{}
\pysigstopsignatures
\sphinxAtStartPar
Calculates and updates the histogram for the specified channel based on the measured data.

\sphinxAtStartPar
The function first checks the selected channel (A, B, C, or D) and retrieves the 
current view range from the corresponding view box. It then creates bins for the 
histogram based on this range, using a total of 61 bins.

\sphinxAtStartPar
Finally, it computes the histogram of the provided data and updates the specified 
curve in the plot with the histogram values.
\begin{quote}\begin{description}
\sphinxlineitem{Parameters}\begin{itemize}
\item {} 
\sphinxAtStartPar
\sphinxstyleliteralstrong{\sphinxupquote{data}} \textendash{} The measured data used to calculate the histogram (list).

\item {} 
\sphinxAtStartPar
\sphinxstyleliteralstrong{\sphinxupquote{curve}} \textendash{} The curve item from the plot in PyQt5 with PyQtGraph that corresponds 
to the channel (PlotCurveItem).

\item {} 
\sphinxAtStartPar
\sphinxstyleliteralstrong{\sphinxupquote{indexChannel}} \textendash{} The identifier of the channel for the histogram, which can 
be “A”, “B”, “C”, or “D” (str).

\end{itemize}

\sphinxlineitem{Returns}
\sphinxAtStartPar
None

\end{description}\end{quote}

\end{fulllineitems}

\index{zoom\_changedA() (StartStopHist.StartStopLogic method)@\spxentry{zoom\_changedA()}\spxextra{StartStopHist.StartStopLogic method}}

\begin{fulllineitems}
\phantomsection\label{\detokenize{StartStopHist:StartStopHist.StartStopLogic.zoom_changedA}}
\pysigstartsignatures
\pysiglinewithargsret{\sphinxbfcode{\sphinxupquote{zoom\_changedA}}}{}{}
\pysigstopsignatures
\sphinxAtStartPar
Automatically calculates and updates the histogram for channel A when the user 
zooms in on the corresponding graph.

\sphinxAtStartPar
The function checks if the zoom was triggered by code or user action. If it was 
triggered by the user, it increments the sentinel for zoom changes. It then 
retrieves the current x\sphinxhyphen{}range from the view box for channel A and computes the 
bin width for the histogram.

\sphinxAtStartPar
Next, it creates 61 bins based on the x\sphinxhyphen{}range and calculates the histogram 
using the measured data. The histogram data is then used to update the curve 
item for channel A in the plot.
\begin{quote}\begin{description}
\sphinxlineitem{Param}
\sphinxAtStartPar
None

\sphinxlineitem{Returns}
\sphinxAtStartPar
None

\end{description}\end{quote}

\end{fulllineitems}

\index{zoom\_changedB() (StartStopHist.StartStopLogic method)@\spxentry{zoom\_changedB()}\spxextra{StartStopHist.StartStopLogic method}}

\begin{fulllineitems}
\phantomsection\label{\detokenize{StartStopHist:StartStopHist.StartStopLogic.zoom_changedB}}
\pysigstartsignatures
\pysiglinewithargsret{\sphinxbfcode{\sphinxupquote{zoom\_changedB}}}{}{}
\pysigstopsignatures
\sphinxAtStartPar
Automatically calculates and updates the histogram for channel B when the user 
zooms in on the corresponding graph.

\sphinxAtStartPar
The function checks if the zoom was triggered by code or user action. If it was 
triggered by the user, it increments the sentinel for zoom changes. It then 
retrieves the current x\sphinxhyphen{}range from the view box for channel B and computes the 
bin width for the histogram.

\sphinxAtStartPar
Next, it creates 61 bins based on the x\sphinxhyphen{}range and calculates the histogram 
using the measured data. The histogram data is then used to update the curve 
item for channel B in the plot.
\begin{quote}\begin{description}
\sphinxlineitem{Param}
\sphinxAtStartPar
None

\sphinxlineitem{Returns}
\sphinxAtStartPar
None

\end{description}\end{quote}

\end{fulllineitems}

\index{zoom\_changedC() (StartStopHist.StartStopLogic method)@\spxentry{zoom\_changedC()}\spxextra{StartStopHist.StartStopLogic method}}

\begin{fulllineitems}
\phantomsection\label{\detokenize{StartStopHist:StartStopHist.StartStopLogic.zoom_changedC}}
\pysigstartsignatures
\pysiglinewithargsret{\sphinxbfcode{\sphinxupquote{zoom\_changedC}}}{}{}
\pysigstopsignatures
\sphinxAtStartPar
Automatically calculates and updates the histogram for channel C when the user 
zooms in on the corresponding graph.

\sphinxAtStartPar
The function checks if the zoom was triggered by code or user action. If it was 
triggered by the user, it increments the sentinel for zoom changes. It then 
retrieves the current x\sphinxhyphen{}range from the view box for channel C and computes the 
bin width for the histogram.

\sphinxAtStartPar
Next, it creates 61 bins based on the x\sphinxhyphen{}range and calculates the histogram 
using the measured data. The histogram data is then used to update the curve 
item for channel C in the plot.
\begin{quote}\begin{description}
\sphinxlineitem{Param}
\sphinxAtStartPar
None

\sphinxlineitem{Returns}
\sphinxAtStartPar
None

\end{description}\end{quote}

\end{fulllineitems}

\index{zoom\_changedD() (StartStopHist.StartStopLogic method)@\spxentry{zoom\_changedD()}\spxextra{StartStopHist.StartStopLogic method}}

\begin{fulllineitems}
\phantomsection\label{\detokenize{StartStopHist:StartStopHist.StartStopLogic.zoom_changedD}}
\pysigstartsignatures
\pysiglinewithargsret{\sphinxbfcode{\sphinxupquote{zoom\_changedD}}}{}{}
\pysigstopsignatures
\sphinxAtStartPar
Automatically calculates and updates the histogram for channel D when the user 
zooms in on the corresponding graph.

\sphinxAtStartPar
The function checks if the zoom was triggered by code or user action. If it was 
triggered by the user, it increments the sentinel for zoom changes. It then 
retrieves the current x\sphinxhyphen{}range from the view box for channel D and computes the 
bin width for the histogram.

\sphinxAtStartPar
Next, it creates 61 bins based on the x\sphinxhyphen{}range and calculates the histogram 
using the measured data. The histogram data is then used to update the curve 
item for channel D in the plot.
\begin{quote}\begin{description}
\sphinxlineitem{Param}
\sphinxAtStartPar
None

\sphinxlineitem{Returns}
\sphinxAtStartPar
None

\end{description}\end{quote}

\end{fulllineitems}


\end{fulllineitems}

\index{WorkerThreadStartStopHistogram (class in StartStopHist)@\spxentry{WorkerThreadStartStopHistogram}\spxextra{class in StartStopHist}}

\begin{fulllineitems}
\phantomsection\label{\detokenize{StartStopHist:StartStopHist.WorkerThreadStartStopHistogram}}
\pysigstartsignatures
\pysiglinewithargsret{\sphinxbfcode{\sphinxupquote{class\DUrole{w,w}{  }}}\sphinxcode{\sphinxupquote{StartStopHist.}}\sphinxbfcode{\sphinxupquote{WorkerThreadStartStopHistogram}}}{\sphinxparam{\DUrole{n,n}{parent}}\sphinxparamcomma \sphinxparam{\DUrole{n,n}{device}}\sphinxparamcomma \sphinxparam{\DUrole{n,n}{sentinelSaveA}}\sphinxparamcomma \sphinxparam{\DUrole{n,n}{sentinelSaveB}}\sphinxparamcomma \sphinxparam{\DUrole{n,n}{sentinelSaveC}}\sphinxparamcomma \sphinxparam{\DUrole{n,n}{sentinelSaveD}}}{}
\pysigstopsignatures
\sphinxAtStartPar
Bases: \sphinxcode{\sphinxupquote{QThread}}

\sphinxAtStartPar
This class represents a worker thread that processes Start\sphinxhyphen{}Stop measurements in a separate thread to avoid blocking the main GUI thread. 
It handles data collection from the Tempico device, processes the data, and emits signals to update the GUI with the results.
\begin{quote}\begin{description}
\sphinxlineitem{Parameters}\begin{itemize}
\item {} 
\sphinxAtStartPar
\sphinxstyleliteralstrong{\sphinxupquote{parent}} \textendash{} The parent QWindow of the thread.

\item {} 
\sphinxAtStartPar
\sphinxstyleliteralstrong{\sphinxupquote{device}} \textendash{} The Tempico device class that handles the measurements.

\item {} 
\sphinxAtStartPar
\sphinxstyleliteralstrong{\sphinxupquote{sentinelSaveD}} (\sphinxstyleliteralemphasis{\sphinxupquote{sentinelSaveA}}\sphinxstyleliteralemphasis{\sphinxupquote{, }}\sphinxstyleliteralemphasis{\sphinxupquote{sentinelSaveB}}\sphinxstyleliteralemphasis{\sphinxupquote{, }}\sphinxstyleliteralemphasis{\sphinxupquote{sentinelSaveC}}\sphinxstyleliteralemphasis{\sphinxupquote{,}}) \textendash{} Boolean flags that determine if data for each channel should be saved.

\end{itemize}

\end{description}\end{quote}
\index{addChannelWarning() (StartStopHist.WorkerThreadStartStopHistogram method)@\spxentry{addChannelWarning()}\spxextra{StartStopHist.WorkerThreadStartStopHistogram method}}

\begin{fulllineitems}
\phantomsection\label{\detokenize{StartStopHist:StartStopHist.WorkerThreadStartStopHistogram.addChannelWarning}}
\pysigstartsignatures
\pysiglinewithargsret{\sphinxbfcode{\sphinxupquote{addChannelWarning}}}{\sphinxparam{\DUrole{n,n}{channel}}}{}
\pysigstopsignatures
\sphinxAtStartPar
Adds the specified channel to the list of channels that need to change mode.
\begin{quote}\begin{description}
\sphinxlineitem{Parameters}
\sphinxAtStartPar
\sphinxstyleliteralstrong{\sphinxupquote{channel}} \textendash{} The channel to be added to the warning list (str).

\sphinxlineitem{Returns}
\sphinxAtStartPar
None

\end{description}\end{quote}

\end{fulllineitems}

\index{changeMaxValue() (StartStopHist.WorkerThreadStartStopHistogram method)@\spxentry{changeMaxValue()}\spxextra{StartStopHist.WorkerThreadStartStopHistogram method}}

\begin{fulllineitems}
\phantomsection\label{\detokenize{StartStopHist:StartStopHist.WorkerThreadStartStopHistogram.changeMaxValue}}
\pysigstartsignatures
\pysiglinewithargsret{\sphinxbfcode{\sphinxupquote{changeMaxValue}}}{\sphinxparam{\DUrole{n,n}{channel}}\sphinxparamcomma \sphinxparam{\DUrole{n,n}{valueMax}}}{}
\pysigstopsignatures
\sphinxAtStartPar
Changes the maximum value for the specified channel.
\begin{quote}\begin{description}
\sphinxlineitem{Parameters}\begin{itemize}
\item {} 
\sphinxAtStartPar
\sphinxstyleliteralstrong{\sphinxupquote{channel}} \textendash{} The channel for which the maximum value is to be changed (str).

\item {} 
\sphinxAtStartPar
\sphinxstyleliteralstrong{\sphinxupquote{valueMax}} \textendash{} The new maximum value to be set (float).

\end{itemize}

\sphinxlineitem{Returns}
\sphinxAtStartPar
None

\end{description}\end{quote}

\end{fulllineitems}

\index{colorValue (StartStopHist.WorkerThreadStartStopHistogram attribute)@\spxentry{colorValue}\spxextra{StartStopHist.WorkerThreadStartStopHistogram attribute}}

\begin{fulllineitems}
\phantomsection\label{\detokenize{StartStopHist:StartStopHist.WorkerThreadStartStopHistogram.colorValue}}
\pysigstartsignatures
\pysigline{\sphinxbfcode{\sphinxupquote{colorValue}}}
\pysigstopsignatures
\end{fulllineitems}

\index{dataPureSignal (StartStopHist.WorkerThreadStartStopHistogram attribute)@\spxentry{dataPureSignal}\spxextra{StartStopHist.WorkerThreadStartStopHistogram attribute}}

\begin{fulllineitems}
\phantomsection\label{\detokenize{StartStopHist:StartStopHist.WorkerThreadStartStopHistogram.dataPureSignal}}
\pysigstartsignatures
\pysigline{\sphinxbfcode{\sphinxupquote{dataPureSignal}}}
\pysigstopsignatures
\end{fulllineitems}

\index{dataSignal (StartStopHist.WorkerThreadStartStopHistogram attribute)@\spxentry{dataSignal}\spxextra{StartStopHist.WorkerThreadStartStopHistogram attribute}}

\begin{fulllineitems}
\phantomsection\label{\detokenize{StartStopHist:StartStopHist.WorkerThreadStartStopHistogram.dataSignal}}
\pysigstartsignatures
\pysigline{\sphinxbfcode{\sphinxupquote{dataSignal}}}
\pysigstopsignatures
\end{fulllineitems}

\index{dialogInit (StartStopHist.WorkerThreadStartStopHistogram attribute)@\spxentry{dialogInit}\spxextra{StartStopHist.WorkerThreadStartStopHistogram attribute}}

\begin{fulllineitems}
\phantomsection\label{\detokenize{StartStopHist:StartStopHist.WorkerThreadStartStopHistogram.dialogInit}}
\pysigstartsignatures
\pysigline{\sphinxbfcode{\sphinxupquote{dialogInit}}}
\pysigstopsignatures
\end{fulllineitems}

\index{dialogIsOpen() (StartStopHist.WorkerThreadStartStopHistogram method)@\spxentry{dialogIsOpen()}\spxextra{StartStopHist.WorkerThreadStartStopHistogram method}}

\begin{fulllineitems}
\phantomsection\label{\detokenize{StartStopHist:StartStopHist.WorkerThreadStartStopHistogram.dialogIsOpen}}
\pysigstartsignatures
\pysiglinewithargsret{\sphinxbfcode{\sphinxupquote{dialogIsOpen}}}{}{}
\pysigstopsignatures
\sphinxAtStartPar
Sets the ‘openDialog’ sentinel to False, indicating that no dialog is open.
\begin{quote}\begin{description}
\sphinxlineitem{Returns}
\sphinxAtStartPar
None

\end{description}\end{quote}

\end{fulllineitems}

\index{dialogSignal (StartStopHist.WorkerThreadStartStopHistogram attribute)@\spxentry{dialogSignal}\spxextra{StartStopHist.WorkerThreadStartStopHistogram attribute}}

\begin{fulllineitems}
\phantomsection\label{\detokenize{StartStopHist:StartStopHist.WorkerThreadStartStopHistogram.dialogSignal}}
\pysigstartsignatures
\pysigline{\sphinxbfcode{\sphinxupquote{dialogSignal}}}
\pysigstopsignatures
\end{fulllineitems}

\index{getNewData() (StartStopHist.WorkerThreadStartStopHistogram method)@\spxentry{getNewData()}\spxextra{StartStopHist.WorkerThreadStartStopHistogram method}}

\begin{fulllineitems}
\phantomsection\label{\detokenize{StartStopHist:StartStopHist.WorkerThreadStartStopHistogram.getNewData}}
\pysigstartsignatures
\pysiglinewithargsret{\sphinxbfcode{\sphinxupquote{getNewData}}}{\sphinxparam{\DUrole{n,n}{channel}}\sphinxparamcomma \sphinxparam{\DUrole{n,n}{channelIndex}}\sphinxparamcomma \sphinxparam{\DUrole{n,n}{stopNumber}}}{}
\pysigstopsignatures
\sphinxAtStartPar
Performs multiple measurements on the device, averages the results, and emits the corresponding data.
If no data is available, it emits None. Increments the total number of measurements taken and activates sentinels
to indicate whether the device is actively collecting data.
\begin{quote}\begin{description}
\sphinxlineitem{Parameters}\begin{itemize}
\item {} 
\sphinxAtStartPar
\sphinxstyleliteralstrong{\sphinxupquote{channel}} \textendash{} The device channel used for measurements (Channel).

\item {} 
\sphinxAtStartPar
\sphinxstyleliteralstrong{\sphinxupquote{channelIndex}} \textendash{} The index of the channel, which can be ‘A’, ‘B’, ‘C’, or ‘D’ (str).

\item {} 
\sphinxAtStartPar
\sphinxstyleliteralstrong{\sphinxupquote{stopNumber}} \textendash{} The stop number to obtain the corresponding measurement (int).

\end{itemize}

\sphinxlineitem{Returns}
\sphinxAtStartPar
The average measurement in milliseconds if data is available; otherwise, returns None (float or None).

\end{description}\end{quote}

\end{fulllineitems}

\index{newMaxValueSignal (StartStopHist.WorkerThreadStartStopHistogram attribute)@\spxentry{newMaxValueSignal}\spxextra{StartStopHist.WorkerThreadStartStopHistogram attribute}}

\begin{fulllineitems}
\phantomsection\label{\detokenize{StartStopHist:StartStopHist.WorkerThreadStartStopHistogram.newMaxValueSignal}}
\pysigstartsignatures
\pysigline{\sphinxbfcode{\sphinxupquote{newMaxValueSignal}}}
\pysigstopsignatures
\end{fulllineitems}

\index{run() (StartStopHist.WorkerThreadStartStopHistogram method)@\spxentry{run()}\spxextra{StartStopHist.WorkerThreadStartStopHistogram method}}

\begin{fulllineitems}
\phantomsection\label{\detokenize{StartStopHist:StartStopHist.WorkerThreadStartStopHistogram.run}}
\pysigstartsignatures
\pysiglinewithargsret{\sphinxbfcode{\sphinxupquote{run}}}{}{}
\pysigstopsignatures
\sphinxAtStartPar
Executes the thread’s main loop to update the graph based on measurements.

\sphinxAtStartPar
The function emits a signal indicating that the thread has been created. It then enters a 
loop that continues as long as the thread is running. Inside the loop, it calls the update 
method to refresh the graph and pauses for 0.5 seconds before the next iteration.
\begin{quote}\begin{description}
\sphinxlineitem{Returns}
\sphinxAtStartPar
None

\end{description}\end{quote}

\end{fulllineitems}

\index{staticMetaObject (StartStopHist.WorkerThreadStartStopHistogram attribute)@\spxentry{staticMetaObject}\spxextra{StartStopHist.WorkerThreadStartStopHistogram attribute}}

\begin{fulllineitems}
\phantomsection\label{\detokenize{StartStopHist:StartStopHist.WorkerThreadStartStopHistogram.staticMetaObject}}
\pysigstartsignatures
\pysigline{\sphinxbfcode{\sphinxupquote{staticMetaObject}}\sphinxbfcode{\sphinxupquote{\DUrole{w,w}{  }\DUrole{p,p}{=}\DUrole{w,w}{  }\textless{}PySide2.QtCore.QMetaObject object\textgreater{}}}}
\pysigstopsignatures
\end{fulllineitems}

\index{stop() (StartStopHist.WorkerThreadStartStopHistogram method)@\spxentry{stop()}\spxextra{StartStopHist.WorkerThreadStartStopHistogram method}}

\begin{fulllineitems}
\phantomsection\label{\detokenize{StartStopHist:StartStopHist.WorkerThreadStartStopHistogram.stop}}
\pysigstartsignatures
\pysiglinewithargsret{\sphinxbfcode{\sphinxupquote{stop}}}{}{}
\pysigstopsignatures
\sphinxAtStartPar
Stops the measurement process and closes the thread.

\sphinxAtStartPar
This method emits a signal to indicate that the thread has been stopped
and sets the running flag to False.
\begin{quote}\begin{description}
\sphinxlineitem{Returns}
\sphinxAtStartPar
None

\end{description}\end{quote}

\end{fulllineitems}

\index{stringValue (StartStopHist.WorkerThreadStartStopHistogram attribute)@\spxentry{stringValue}\spxextra{StartStopHist.WorkerThreadStartStopHistogram attribute}}

\begin{fulllineitems}
\phantomsection\label{\detokenize{StartStopHist:StartStopHist.WorkerThreadStartStopHistogram.stringValue}}
\pysigstartsignatures
\pysigline{\sphinxbfcode{\sphinxupquote{stringValue}}}
\pysigstopsignatures
\end{fulllineitems}

\index{threadCreated (StartStopHist.WorkerThreadStartStopHistogram attribute)@\spxentry{threadCreated}\spxextra{StartStopHist.WorkerThreadStartStopHistogram attribute}}

\begin{fulllineitems}
\phantomsection\label{\detokenize{StartStopHist:StartStopHist.WorkerThreadStartStopHistogram.threadCreated}}
\pysigstartsignatures
\pysigline{\sphinxbfcode{\sphinxupquote{threadCreated}}}
\pysigstopsignatures
\end{fulllineitems}

\index{update() (StartStopHist.WorkerThreadStartStopHistogram method)@\spxentry{update()}\spxextra{StartStopHist.WorkerThreadStartStopHistogram method}}

\begin{fulllineitems}
\phantomsection\label{\detokenize{StartStopHist:StartStopHist.WorkerThreadStartStopHistogram.update}}
\pysigstartsignatures
\pysiglinewithargsret{\sphinxbfcode{\sphinxupquote{update}}}{}{}
\pysigstopsignatures
\sphinxAtStartPar
Performs measurements and updates the graph with the new data.

\sphinxAtStartPar
The function verifies the device connection and retrieves measurements for each channel (A, B, C, D). 
It checks if the majority of measurements are out of range when in mode 1. If so, it emits a signal 
to open a dialog box asking the user if they want to change the mode. If there are no measurements, 
the function prioritizes the alert to change mode. Otherwise, it emits the measurement value to update 
the graph.
\begin{quote}\begin{description}
\sphinxlineitem{Returns}
\sphinxAtStartPar
None

\end{description}\end{quote}

\end{fulllineitems}


\end{fulllineitems}


\sphinxstepscope


\chapter{aboutDialog module}
\label{\detokenize{aboutDialog:module-aboutDialog}}\label{\detokenize{aboutDialog:aboutdialog-module}}\label{\detokenize{aboutDialog::doc}}\index{module@\spxentry{module}!aboutDialog@\spxentry{aboutDialog}}\index{aboutDialog@\spxentry{aboutDialog}!module@\spxentry{module}}\index{Ui\_AboutDialog (class in aboutDialog)@\spxentry{Ui\_AboutDialog}\spxextra{class in aboutDialog}}

\begin{fulllineitems}
\phantomsection\label{\detokenize{aboutDialog:aboutDialog.Ui_AboutDialog}}
\pysigstartsignatures
\pysigline{\sphinxbfcode{\sphinxupquote{class\DUrole{w,w}{  }}}\sphinxcode{\sphinxupquote{aboutDialog.}}\sphinxbfcode{\sphinxupquote{Ui\_AboutDialog}}}
\pysigstopsignatures
\sphinxAtStartPar
Bases: \sphinxcode{\sphinxupquote{object}}
\index{acceptButton() (aboutDialog.Ui\_AboutDialog method)@\spxentry{acceptButton()}\spxextra{aboutDialog.Ui\_AboutDialog method}}

\begin{fulllineitems}
\phantomsection\label{\detokenize{aboutDialog:aboutDialog.Ui_AboutDialog.acceptButton}}
\pysigstartsignatures
\pysiglinewithargsret{\sphinxbfcode{\sphinxupquote{acceptButton}}}{}{}
\pysigstopsignatures
\end{fulllineitems}

\index{open\_link() (aboutDialog.Ui\_AboutDialog method)@\spxentry{open\_link()}\spxextra{aboutDialog.Ui\_AboutDialog method}}

\begin{fulllineitems}
\phantomsection\label{\detokenize{aboutDialog:aboutDialog.Ui_AboutDialog.open_link}}
\pysigstartsignatures
\pysiglinewithargsret{\sphinxbfcode{\sphinxupquote{open\_link}}}{\sphinxparam{\DUrole{n,n}{url}}}{}
\pysigstopsignatures
\sphinxAtStartPar
Opens a specified URL in the default web browser.

\sphinxAtStartPar
This function is triggered when a link is clicked and opens the provided URL 
using the system’s default web browser.
\begin{quote}\begin{description}
\sphinxlineitem{Parameters}
\sphinxAtStartPar
\sphinxstyleliteralstrong{\sphinxupquote{link}} (\sphinxstyleliteralemphasis{\sphinxupquote{str}}) \textendash{} The URL to be opened.

\sphinxlineitem{Returns}
\sphinxAtStartPar
None

\end{description}\end{quote}

\end{fulllineitems}

\index{setupUi() (aboutDialog.Ui\_AboutDialog method)@\spxentry{setupUi()}\spxextra{aboutDialog.Ui\_AboutDialog method}}

\begin{fulllineitems}
\phantomsection\label{\detokenize{aboutDialog:aboutDialog.Ui_AboutDialog.setupUi}}
\pysigstartsignatures
\pysiglinewithargsret{\sphinxbfcode{\sphinxupquote{setupUi}}}{\sphinxparam{\DUrole{n,n}{AboutDialog}}}{}
\pysigstopsignatures
\end{fulllineitems}


\end{fulllineitems}


\sphinxstepscope


\chapter{createsavefile module}
\label{\detokenize{createsavefile:module-createsavefile}}\label{\detokenize{createsavefile:createsavefile-module}}\label{\detokenize{createsavefile::doc}}\index{module@\spxentry{module}!createsavefile@\spxentry{createsavefile}}\index{createsavefile@\spxentry{createsavefile}!module@\spxentry{module}}\index{createsavefile (class in createsavefile)@\spxentry{createsavefile}\spxextra{class in createsavefile}}

\begin{fulllineitems}
\phantomsection\label{\detokenize{createsavefile:createsavefile.createsavefile}}
\pysigstartsignatures
\pysigline{\sphinxbfcode{\sphinxupquote{class\DUrole{w,w}{  }}}\sphinxcode{\sphinxupquote{createsavefile.}}\sphinxbfcode{\sphinxupquote{createsavefile}}}
\pysigstopsignatures
\sphinxAtStartPar
Bases: \sphinxcode{\sphinxupquote{object}}
\index{check\_folder\_and\_file() (createsavefile.createsavefile method)@\spxentry{check\_folder\_and\_file()}\spxextra{createsavefile.createsavefile method}}

\begin{fulllineitems}
\phantomsection\label{\detokenize{createsavefile:createsavefile.createsavefile.check_folder_and_file}}
\pysigstartsignatures
\pysiglinewithargsret{\sphinxbfcode{\sphinxupquote{check\_folder\_and\_file}}}{}{}
\pysigstopsignatures
\sphinxAtStartPar
Checks whether the folder ‘TempicoSoftwareData’ and the file ‘data\_constants.txt’ 
exist in the user’s ‘Documents’ directory.

\sphinxAtStartPar
The function constructs the paths for both the folder and the file, and checks if both exist:
\sphinxhyphen{} Folder path: \textless{}Documents\textgreater{}/TempicoSoftwareData
\sphinxhyphen{} File name: data\_constants.txt

\sphinxAtStartPar
If both the folder and the file exist, the function returns True. Otherwise, it returns False.
\begin{quote}\begin{description}
\sphinxlineitem{Returns}
\sphinxAtStartPar
bool: True if the folder and file exist, False otherwise.

\end{description}\end{quote}

\end{fulllineitems}

\index{create\_folder() (createsavefile.createsavefile method)@\spxentry{create\_folder()}\spxextra{createsavefile.createsavefile method}}

\begin{fulllineitems}
\phantomsection\label{\detokenize{createsavefile:createsavefile.createsavefile.create_folder}}
\pysigstartsignatures
\pysiglinewithargsret{\sphinxbfcode{\sphinxupquote{create\_folder}}}{}{}
\pysigstopsignatures
\sphinxAtStartPar
Checks if the specified folder and file already exist. If they do exist, it creates 
the folder and file by calling the ‘create\_folder\_and\_file’ function.
\begin{quote}\begin{description}
\sphinxlineitem{Returns}
\sphinxAtStartPar
None

\end{description}\end{quote}

\end{fulllineitems}

\index{create\_folder\_and\_file() (createsavefile.createsavefile method)@\spxentry{create\_folder\_and\_file()}\spxextra{createsavefile.createsavefile method}}

\begin{fulllineitems}
\phantomsection\label{\detokenize{createsavefile:createsavefile.createsavefile.create_folder_and_file}}
\pysigstartsignatures
\pysiglinewithargsret{\sphinxbfcode{\sphinxupquote{create\_folder\_and\_file}}}{}{}
\pysigstopsignatures
\sphinxAtStartPar
Creates a folder named ‘TempicoSoftwareData’ inside the user’s ‘Documents’ directory, 
and creates a file named ‘data\_constants.txt’ (or ‘.data\_constants.txt’ on Unix\sphinxhyphen{}based systems) inside that folder. 
The file is populated with predefined default values for histogram and data names. 
The file is hidden depending on the operating system.

\sphinxAtStartPar
The folder and file are created with the following details:
\sphinxhyphen{} Folder path: \textless{}Documents\textgreater{}/TempicoSoftwareData
\sphinxhyphen{} File name: data\_constants.txt (or ‘.data\_constants.txt’ for Unix\sphinxhyphen{}based systems)
\sphinxhyphen{} The file contains the following default data:
\sphinxhyphen{} Folder Path
\sphinxhyphen{} Default Histogram Name
\sphinxhyphen{} Default g2 Name
\sphinxhyphen{} Default Lifetime Name

\sphinxAtStartPar
In case of a Windows system, the file is made hidden using the Windows API. For Unix\sphinxhyphen{}based systems (Linux, macOS), 
the file is renamed to be hidden by adding a dot (‘.’) before the filename.
\begin{quote}\begin{description}
\sphinxlineitem{Raises}
\sphinxAtStartPar
\sphinxstyleliteralstrong{\sphinxupquote{OSError}} \textendash{} If an error occurs during folder or file creation, or when accessing the filesystem.

\sphinxlineitem{Returns}
\sphinxAtStartPar
None

\end{description}\end{quote}

\end{fulllineitems}

\index{read\_default\_data() (createsavefile.createsavefile method)@\spxentry{read\_default\_data()}\spxextra{createsavefile.createsavefile method}}

\begin{fulllineitems}
\phantomsection\label{\detokenize{createsavefile:createsavefile.createsavefile.read_default_data}}
\pysigstartsignatures
\pysiglinewithargsret{\sphinxbfcode{\sphinxupquote{read\_default\_data}}}{}{}
\pysigstopsignatures
\sphinxAtStartPar
Reads the default data from the file ‘data\_constants.txt’ located in the 
‘TempicoSoftwareData’ folder inside the user’s ‘Documents’ directory.

\sphinxAtStartPar
This function attempts to read the file line by line and extracts key\sphinxhyphen{}value pairs 
formatted as “key: value”. The first line, which contains the folder path, is 
processed separately. The extracted key\sphinxhyphen{}value pairs are stored in a dictionary, 
which is returned as the output.

\sphinxAtStartPar
If the file is not found or an error occurs while reading the file, the function
returns None.
\begin{quote}\begin{description}
\sphinxlineitem{Returns}
\sphinxAtStartPar
dict or None: A dictionary containing the data from the file, or None if 
an error occurred.

\end{description}\end{quote}

\end{fulllineitems}

\index{save\_FLIM\_data() (createsavefile.createsavefile method)@\spxentry{save\_FLIM\_data()}\spxextra{createsavefile.createsavefile method}}

\begin{fulllineitems}
\phantomsection\label{\detokenize{createsavefile:createsavefile.createsavefile.save_FLIM_data}}
\pysigstartsignatures
\pysiglinewithargsret{\sphinxbfcode{\sphinxupquote{save\_FLIM\_data}}}{\sphinxparam{\DUrole{n,n}{data}}\sphinxparamcomma \sphinxparam{\DUrole{n,n}{file\_name}}\sphinxparamcomma \sphinxparam{\DUrole{n,n}{folder\_path}}\sphinxparamcomma \sphinxparam{\DUrole{n,n}{settings}}\sphinxparamcomma \sphinxparam{\DUrole{n,n}{extension}}\sphinxparamcomma \sphinxparam{\DUrole{n,n}{textLabel}}}{}
\pysigstopsignatures
\sphinxAtStartPar
Saves FLIM data (time and FLIM values) into a text file in a specified folder. The function
ensures that the provided time and FLIM values have the same length and writes them into 
a file along with specified settings and a label for the FLIM values.

\sphinxAtStartPar
The file is saved in the specified folder path, with the provided file name and extension.
\begin{quote}\begin{description}
\sphinxlineitem{Parameters}\begin{itemize}
\item {} 
\sphinxAtStartPar
\sphinxstyleliteralstrong{\sphinxupquote{data}} \textendash{} A tuple where the first element is a list of time values and the second 
element is a list of corresponding FLIM values (tuple of lists).

\item {} 
\sphinxAtStartPar
\sphinxstyleliteralstrong{\sphinxupquote{file\_name}} \textendash{} The name of the output file (str).

\item {} 
\sphinxAtStartPar
\sphinxstyleliteralstrong{\sphinxupquote{folder\_path}} \textendash{} The path to the folder where the file will be saved (str).

\item {} 
\sphinxAtStartPar
\sphinxstyleliteralstrong{\sphinxupquote{settings}} \textendash{} A string representing the settings to be written in the first line of the file (str).

\item {} 
\sphinxAtStartPar
\sphinxstyleliteralstrong{\sphinxupquote{extension}} \textendash{} The file extension for the output file (str).

\item {} 
\sphinxAtStartPar
\sphinxstyleliteralstrong{\sphinxupquote{textLabel}} \textendash{} A label to be written before the FLIM values in the file (str).

\end{itemize}

\sphinxlineitem{Raises}
\sphinxAtStartPar
\sphinxstyleliteralstrong{\sphinxupquote{ValueError}} \textendash{} If the lengths of the time and FLIM value lists do not match.

\sphinxlineitem{Returns}
\sphinxAtStartPar
None

\end{description}\end{quote}

\end{fulllineitems}

\index{save\_g2\_data() (createsavefile.createsavefile method)@\spxentry{save\_g2\_data()}\spxextra{createsavefile.createsavefile method}}

\begin{fulllineitems}
\phantomsection\label{\detokenize{createsavefile:createsavefile.createsavefile.save_g2_data}}
\pysigstartsignatures
\pysiglinewithargsret{\sphinxbfcode{\sphinxupquote{save\_g2\_data}}}{\sphinxparam{\DUrole{n,n}{data}}\sphinxparamcomma \sphinxparam{\DUrole{n,n}{file\_name}}\sphinxparamcomma \sphinxparam{\DUrole{n,n}{folder\_path}}\sphinxparamcomma \sphinxparam{\DUrole{n,n}{settings}}\sphinxparamcomma \sphinxparam{\DUrole{n,n}{extension}}\sphinxparamcomma \sphinxparam{\DUrole{n,n}{textLabel}}}{}
\pysigstopsignatures
\sphinxAtStartPar
Saves g2 data (tau and g2 values) into a text file in a specified folder. The function
ensures that the provided tau and g2 values have the same length and writes them into 
a file along with specified settings and a label for the g2 values.

\sphinxAtStartPar
The file is saved in the specified folder path, with the provided file name and extension.
\begin{quote}\begin{description}
\sphinxlineitem{Parameters}\begin{itemize}
\item {} 
\sphinxAtStartPar
\sphinxstyleliteralstrong{\sphinxupquote{data}} \textendash{} A tuple where the first element is a list of tau values and the second 
element is a list of corresponding g2 values (tuple of lists).

\item {} 
\sphinxAtStartPar
\sphinxstyleliteralstrong{\sphinxupquote{file\_name}} \textendash{} The name of the output file (str).

\item {} 
\sphinxAtStartPar
\sphinxstyleliteralstrong{\sphinxupquote{folder\_path}} \textendash{} The path to the folder where the file will be saved (str).

\item {} 
\sphinxAtStartPar
\sphinxstyleliteralstrong{\sphinxupquote{settings}} \textendash{} A string representing the settings to be written in the first line of the file (str).

\item {} 
\sphinxAtStartPar
\sphinxstyleliteralstrong{\sphinxupquote{extension}} \textendash{} The file extension for the output file (str).

\item {} 
\sphinxAtStartPar
\sphinxstyleliteralstrong{\sphinxupquote{textLabel}} \textendash{} A label to be written before the g2 values in the file (str).

\end{itemize}

\sphinxlineitem{Raises}
\sphinxAtStartPar
\sphinxstyleliteralstrong{\sphinxupquote{ValueError}} \textendash{} If the lengths of the tau and g2 value lists do not match.

\sphinxlineitem{Returns}
\sphinxAtStartPar
None

\end{description}\end{quote}

\end{fulllineitems}

\index{save\_lists\_as\_columns\_txt() (createsavefile.createsavefile method)@\spxentry{save\_lists\_as\_columns\_txt()}\spxextra{createsavefile.createsavefile method}}

\begin{fulllineitems}
\phantomsection\label{\detokenize{createsavefile:createsavefile.createsavefile.save_lists_as_columns_txt}}
\pysigstartsignatures
\pysiglinewithargsret{\sphinxbfcode{\sphinxupquote{save\_lists\_as\_columns\_txt}}}{\sphinxparam{\DUrole{n,n}{data\_lists}}\sphinxparamcomma \sphinxparam{\DUrole{n,n}{file\_names}}\sphinxparamcomma \sphinxparam{\DUrole{n,n}{column\_names}}\sphinxparamcomma \sphinxparam{\DUrole{n,n}{path}}\sphinxparamcomma \sphinxparam{\DUrole{n,n}{settings}}\sphinxparamcomma \sphinxparam{\DUrole{n,n}{extension}}}{}
\pysigstopsignatures
\sphinxAtStartPar
Saves multiple lists of data as separate text files, where each list is written as a 
column in the text file. The files are saved in the specified directory with the 
provided file names, and the columns are named based on the given column names.

\sphinxAtStartPar
This function ensures that the specified directory exists, and raises an error if 
the lengths of the input lists do not match.
\begin{quote}\begin{description}
\sphinxlineitem{Parameters}\begin{itemize}
\item {} 
\sphinxAtStartPar
\sphinxstyleliteralstrong{\sphinxupquote{data\_lists}} \textendash{} A list of lists, where each list contains the data to be written to a file (list).

\item {} 
\sphinxAtStartPar
\sphinxstyleliteralstrong{\sphinxupquote{file\_names}} \textendash{} A list of strings specifying the names of the output files (list of str).

\item {} 
\sphinxAtStartPar
\sphinxstyleliteralstrong{\sphinxupquote{column\_names}} \textendash{} A list of strings specifying the names of the columns in the text files (list of str).

\item {} 
\sphinxAtStartPar
\sphinxstyleliteralstrong{\sphinxupquote{path}} \textendash{} The directory path where the files will be saved (str).

\item {} 
\sphinxAtStartPar
\sphinxstyleliteralstrong{\sphinxupquote{settings}} \textendash{} A list of settings to be written as the first line in each file (list of str).

\item {} 
\sphinxAtStartPar
\sphinxstyleliteralstrong{\sphinxupquote{extension}} \textendash{} The file extension for the output files (str).

\end{itemize}

\sphinxlineitem{Raises}
\sphinxAtStartPar
\sphinxstyleliteralstrong{\sphinxupquote{ValueError}} \textendash{} If the lengths of data\_lists, file\_names, and column\_names do not match.

\sphinxlineitem{Returns}
\sphinxAtStartPar
None

\end{description}\end{quote}

\end{fulllineitems}


\end{fulllineitems}


\sphinxstepscope


\chapter{findDevices module}
\label{\detokenize{findDevices:module-findDevices}}\label{\detokenize{findDevices:finddevices-module}}\label{\detokenize{findDevices::doc}}\index{module@\spxentry{module}!findDevices@\spxentry{findDevices}}\index{findDevices@\spxentry{findDevices}!module@\spxentry{module}}\index{PyTempicoManager (class in findDevices)@\spxentry{PyTempicoManager}\spxextra{class in findDevices}}

\begin{fulllineitems}
\phantomsection\label{\detokenize{findDevices:findDevices.PyTempicoManager}}
\pysigstartsignatures
\pysigline{\sphinxbfcode{\sphinxupquote{class\DUrole{w,w}{  }}}\sphinxcode{\sphinxupquote{findDevices.}}\sphinxbfcode{\sphinxupquote{PyTempicoManager}}}
\pysigstopsignatures
\sphinxAtStartPar
Bases: \sphinxcode{\sphinxupquote{object}}

\sphinxAtStartPar
Class for managing the search and connection of Tempico devices connected to serial ports on the PC.

\sphinxAtStartPar
This class is responsible for scanning the available serial ports on the system and searching for Tempico devices 
connected to those ports. It facilitates interaction with Tempico devices by allowing their identification and connection.
\begin{quote}\begin{description}
\sphinxlineitem{Parameters}
\sphinxAtStartPar
\sphinxstyleliteralstrong{\sphinxupquote{None}} \textendash{} No parameters are required for the initialization of the class.

\sphinxlineitem{Returns}
\sphinxAtStartPar
None

\end{description}\end{quote}
\index{find\_devices() (findDevices.PyTempicoManager method)@\spxentry{find\_devices()}\spxextra{findDevices.PyTempicoManager method}}

\begin{fulllineitems}
\phantomsection\label{\detokenize{findDevices:findDevices.PyTempicoManager.find_devices}}
\pysigstartsignatures
\pysiglinewithargsret{\sphinxbfcode{\sphinxupquote{find\_devices}}}{\sphinxparam{\DUrole{n,n}{vid\_s}}\sphinxparamcomma \sphinxparam{\DUrole{n,n}{pid\_s}}}{}
\pysigstopsignatures
\sphinxAtStartPar
Finds and verifies whether a device with the given VID and PID is a Tempico device.

\sphinxAtStartPar
This function takes the Vendor ID (VID) and Product ID (PID) as inputs, converts them to integers, 
and attempts to open the device using these values. It then checks if the manufacturer and product 
strings match the expected values for a Tempico device.
\begin{quote}\begin{description}
\sphinxlineitem{Parameters}\begin{itemize}
\item {} 
\sphinxAtStartPar
\sphinxstyleliteralstrong{\sphinxupquote{vid\_s}} (\sphinxstyleliteralemphasis{\sphinxupquote{str}}) \textendash{} The Vendor ID (VID) of the device in string format.

\item {} 
\sphinxAtStartPar
\sphinxstyleliteralstrong{\sphinxupquote{pid\_s}} (\sphinxstyleliteralemphasis{\sphinxupquote{str}}) \textendash{} The Product ID (PID) of the device in string format.

\end{itemize}

\sphinxlineitem{Returns}
\sphinxAtStartPar
\sphinxtitleref{True} if the device is a Tempico, \sphinxtitleref{False} otherwise.

\sphinxlineitem{Return type}
\sphinxAtStartPar
bool

\end{description}\end{quote}

\end{fulllineitems}

\index{get\_pytempico\_devices() (findDevices.PyTempicoManager method)@\spxentry{get\_pytempico\_devices()}\spxextra{findDevices.PyTempicoManager method}}

\begin{fulllineitems}
\phantomsection\label{\detokenize{findDevices:findDevices.PyTempicoManager.get_pytempico_devices}}
\pysigstartsignatures
\pysiglinewithargsret{\sphinxbfcode{\sphinxupquote{get\_pytempico\_devices}}}{}{}
\pysigstopsignatures
\sphinxAtStartPar
Searches for Tempico devices among the available serial ports and returns a list of their port names.

\sphinxAtStartPar
This function scans the system’s serial ports, checks if they correspond to a Tempico device by
retrieving the VID and PID, and verifies if they match the Tempico device specifications.
Bluetooth devices are excluded from the search.
\begin{quote}\begin{description}
\sphinxlineitem{Returns}
\sphinxAtStartPar
A list of port names corresponding to Tempico devices.

\sphinxlineitem{Return type}
\sphinxAtStartPar
list of str

\end{description}\end{quote}

\end{fulllineitems}

\index{get\_vid\_pid() (findDevices.PyTempicoManager method)@\spxentry{get\_vid\_pid()}\spxextra{findDevices.PyTempicoManager method}}

\begin{fulllineitems}
\phantomsection\label{\detokenize{findDevices:findDevices.PyTempicoManager.get_vid_pid}}
\pysigstartsignatures
\pysiglinewithargsret{\sphinxbfcode{\sphinxupquote{get\_vid\_pid}}}{\sphinxparam{\DUrole{n,n}{vid\_pid\_information}}}{}
\pysigstopsignatures
\sphinxAtStartPar
Extracts the Vendor ID (VID) and Product ID (PID) from a string and returns them as a tuple.

\sphinxAtStartPar
This function processes a string that contains the VID and PID information in the format 
‘VID:PID=xxxx:yyyy’. It splits the string and retrieves the VID and PID values, returning 
them as a tuple of strings.
\begin{quote}\begin{description}
\sphinxlineitem{Parameters}
\sphinxAtStartPar
\sphinxstyleliteralstrong{\sphinxupquote{vid\_pid\_information}} (\sphinxstyleliteralemphasis{\sphinxupquote{str}}) \textendash{} A string containing the VID and PID information.

\sphinxlineitem{Returns}
\sphinxAtStartPar
A tuple containing the VID and PID as strings (vid, pid).

\sphinxlineitem{Return type}
\sphinxAtStartPar
tuple

\end{description}\end{quote}

\end{fulllineitems}

\index{verify\_pyTempico() (findDevices.PyTempicoManager method)@\spxentry{verify\_pyTempico()}\spxextra{findDevices.PyTempicoManager method}}

\begin{fulllineitems}
\phantomsection\label{\detokenize{findDevices:findDevices.PyTempicoManager.verify_pyTempico}}
\pysigstartsignatures
\pysiglinewithargsret{\sphinxbfcode{\sphinxupquote{verify\_pyTempico}}}{\sphinxparam{\DUrole{n,n}{tuple\_vid\_pid}}}{}
\pysigstopsignatures
\sphinxAtStartPar
Verifies whether the connected device is a Tempico device.

\sphinxAtStartPar
This function checks if the device’s Vendor ID (VID) and Product ID (PID) match the values 
corresponding to a Tempico device. It returns \sphinxtitleref{True} if the device is identified as a Tempico, 
and \sphinxtitleref{False} otherwise.
\begin{quote}\begin{description}
\sphinxlineitem{Parameters}
\sphinxAtStartPar
\sphinxstyleliteralstrong{\sphinxupquote{tuple\_vid\_pid}} (\sphinxstyleliteralemphasis{\sphinxupquote{tuple}}) \textendash{} A tuple containing the VID and PID of the device.

\sphinxlineitem{Returns}
\sphinxAtStartPar
\sphinxtitleref{True} if the device is a Tempico, \sphinxtitleref{False} otherwise.

\sphinxlineitem{Return type}
\sphinxAtStartPar
bool

\end{description}\end{quote}

\end{fulllineitems}


\end{fulllineitems}


\sphinxstepscope


\chapter{generalsettings module}
\label{\detokenize{generalsettings:module-generalsettings}}\label{\detokenize{generalsettings:generalsettings-module}}\label{\detokenize{generalsettings::doc}}\index{module@\spxentry{module}!generalsettings@\spxentry{generalsettings}}\index{generalsettings@\spxentry{generalsettings}!module@\spxentry{module}}\index{GeneralSettingsWindow (class in generalsettings)@\spxentry{GeneralSettingsWindow}\spxextra{class in generalsettings}}

\begin{fulllineitems}
\phantomsection\label{\detokenize{generalsettings:generalsettings.GeneralSettingsWindow}}
\pysigstartsignatures
\pysiglinewithargsret{\sphinxbfcode{\sphinxupquote{class\DUrole{w,w}{  }}}\sphinxcode{\sphinxupquote{generalsettings.}}\sphinxbfcode{\sphinxupquote{GeneralSettingsWindow}}}{\sphinxparam{\DUrole{n,n}{device}}}{}
\pysigstopsignatures
\sphinxAtStartPar
Bases: \sphinxcode{\sphinxupquote{QDialog}}

\sphinxAtStartPar
Represents the General Settings window for the Tempico device.

\sphinxAtStartPar
This class creates a dialog window that allows users to configure general settings that affect all channels (A, B, C, D) and the start settings of the Tempico device. It provides a user interface to adjust various parameters applicable across the entire device.
\begin{quote}\begin{description}
\sphinxlineitem{Parameters}
\sphinxAtStartPar
\sphinxstyleliteralstrong{\sphinxupquote{device}} (\sphinxstyleliteralemphasis{\sphinxupquote{object}}) \textendash{} The Tempico device instance that the settings will apply to.

\end{description}\end{quote}
\index{event() (generalsettings.GeneralSettingsWindow method)@\spxentry{event()}\spxextra{generalsettings.GeneralSettingsWindow method}}

\begin{fulllineitems}
\phantomsection\label{\detokenize{generalsettings:generalsettings.GeneralSettingsWindow.event}}
\pysigstartsignatures
\pysiglinewithargsret{\sphinxbfcode{\sphinxupquote{event}}}{\sphinxparam{\DUrole{n,n}{event}}}{}
\pysigstopsignatures
\sphinxAtStartPar
Handles the event when the “?” (What’s This) help button is clicked.

\sphinxAtStartPar
This function intercepts the event triggered when the user clicks the “?” button (entering What’s This mode). 
It exits What’s This mode immediately, changes the mouse cursor back to the arrow, and displays a help window with relevant information.
\begin{quote}\begin{description}
\sphinxlineitem{Parameters}
\sphinxAtStartPar
\sphinxstyleliteralstrong{\sphinxupquote{event}} (\sphinxstyleliteralemphasis{\sphinxupquote{QEvent}}) \textendash{} The event object representing the triggered event.

\sphinxlineitem{Returns}
\sphinxAtStartPar
True if the event is handled; otherwise, it passes the event to the parent class for default processing.

\sphinxlineitem{Return type}
\sphinxAtStartPar
bool

\end{description}\end{quote}

\end{fulllineitems}

\index{getsettings() (generalsettings.GeneralSettingsWindow method)@\spxentry{getsettings()}\spxextra{generalsettings.GeneralSettingsWindow method}}

\begin{fulllineitems}
\phantomsection\label{\detokenize{generalsettings:generalsettings.GeneralSettingsWindow.getsettings}}
\pysigstartsignatures
\pysiglinewithargsret{\sphinxbfcode{\sphinxupquote{getsettings}}}{}{}
\pysigstopsignatures
\sphinxAtStartPar
Retrieves the device’s current settings, such as the number of runs and threshold voltage, 
and updates the corresponding values in the graphical interface.

\sphinxAtStartPar
The function fetches the number of runs and threshold voltage from the connected device, 
and then sets these values in the dialog’s spinbox and combobox in the user interface.
\begin{quote}\begin{description}
\sphinxlineitem{Returns}
\sphinxAtStartPar
None

\end{description}\end{quote}

\end{fulllineitems}

\index{setsettings() (generalsettings.GeneralSettingsWindow method)@\spxentry{setsettings()}\spxextra{generalsettings.GeneralSettingsWindow method}}

\begin{fulllineitems}
\phantomsection\label{\detokenize{generalsettings:generalsettings.GeneralSettingsWindow.setsettings}}
\pysigstartsignatures
\pysiglinewithargsret{\sphinxbfcode{\sphinxupquote{setsettings}}}{}{}
\pysigstopsignatures
\sphinxAtStartPar
Applies the settings provided by the user to the connected device.

\sphinxAtStartPar
The function retrieves the user\sphinxhyphen{}defined values for the number of runs and threshold voltage 
from the graphical interface and sends these values to the device. After applying the settings, 
the function closes the settings dialog.
\begin{quote}\begin{description}
\sphinxlineitem{Returns}
\sphinxAtStartPar
None

\end{description}\end{quote}

\end{fulllineitems}

\index{showHelp() (generalsettings.GeneralSettingsWindow method)@\spxentry{showHelp()}\spxextra{generalsettings.GeneralSettingsWindow method}}

\begin{fulllineitems}
\phantomsection\label{\detokenize{generalsettings:generalsettings.GeneralSettingsWindow.showHelp}}
\pysigstartsignatures
\pysiglinewithargsret{\sphinxbfcode{\sphinxupquote{showHelp}}}{}{}
\pysigstopsignatures
\sphinxAtStartPar
Displays a help dialog with information about general settings.

\sphinxAtStartPar
This function is triggered when the “?” (What’s This) help button is clicked. It creates and displays a message box containing detailed information about the general settings, including instructions for the threshold voltage and the number of runs.

\sphinxAtStartPar
The message box provides the following details:
\sphinxhyphen{} \sphinxstylestrong{Threshold voltage}: Specifies the acceptable range for input values (0.60 V to 1.60 V).
\sphinxhyphen{} \sphinxstylestrong{Number of runs}: Describes the number of measurements performed in each channel during one data collection.
\begin{quote}\begin{description}
\sphinxlineitem{Returns}
\sphinxAtStartPar
None

\end{description}\end{quote}

\end{fulllineitems}

\index{staticMetaObject (generalsettings.GeneralSettingsWindow attribute)@\spxentry{staticMetaObject}\spxextra{generalsettings.GeneralSettingsWindow attribute}}

\begin{fulllineitems}
\phantomsection\label{\detokenize{generalsettings:generalsettings.GeneralSettingsWindow.staticMetaObject}}
\pysigstartsignatures
\pysigline{\sphinxbfcode{\sphinxupquote{staticMetaObject}}\sphinxbfcode{\sphinxupquote{\DUrole{w,w}{  }\DUrole{p,p}{=}\DUrole{w,w}{  }\textless{}PySide2.QtCore.QMetaObject object\textgreater{}}}}
\pysigstopsignatures
\end{fulllineitems}


\end{fulllineitems}


\sphinxstepscope


\chapter{main module}
\label{\detokenize{main:module-main}}\label{\detokenize{main:main-module}}\label{\detokenize{main::doc}}\index{module@\spxentry{module}!main@\spxentry{main}}\index{main@\spxentry{main}!module@\spxentry{module}}\index{MainWindow (class in main)@\spxentry{MainWindow}\spxextra{class in main}}

\begin{fulllineitems}
\phantomsection\label{\detokenize{main:main.MainWindow}}
\pysigstartsignatures
\pysiglinewithargsret{\sphinxbfcode{\sphinxupquote{class\DUrole{w,w}{  }}}\sphinxcode{\sphinxupquote{main.}}\sphinxbfcode{\sphinxupquote{MainWindow}}}{\sphinxparam{\DUrole{n,n}{parent}\DUrole{o,o}{=}\DUrole{default_value}{None}}\sphinxparamcomma \sphinxparam{\DUrole{o,o}{*}\DUrole{n,n}{args}}}{}
\pysigstopsignatures
\sphinxAtStartPar
Bases: \sphinxcode{\sphinxupquote{QMainWindow}}

\sphinxAtStartPar
Main application window class.

\sphinxAtStartPar
This class is responsible for creating the main window of the application, including its tabs and graphical user interface (GUI) components. 
It serves as a bridge between the UI classes that handle the design elements and the logic classes that manage the application’s functionalities.

\sphinxAtStartPar
The main tasks of this class include:
\sphinxhyphen{} Initializing the main window and its layout.
\sphinxhyphen{} Creating and managing tabs for different sections of the application.
\sphinxhyphen{} Integrating UI design elements with the logical operations for each functionality.
\begin{quote}\begin{description}
\sphinxlineitem{Parameters}\begin{itemize}
\item {} 
\sphinxAtStartPar
\sphinxstyleliteralstrong{\sphinxupquote{parent}} (\sphinxstyleliteralemphasis{\sphinxupquote{QWidget}}\sphinxstyleliteralemphasis{\sphinxupquote{, }}\sphinxstyleliteralemphasis{\sphinxupquote{optional}}) \textendash{} The parent widget (optional).

\item {} 
\sphinxAtStartPar
\sphinxstyleliteralstrong{\sphinxupquote{args}} (\sphinxstyleliteralemphasis{\sphinxupquote{tuple}}) \textendash{} Additional arguments.

\end{itemize}

\end{description}\end{quote}
\index{Helpg2Button() (main.MainWindow method)@\spxentry{Helpg2Button()}\spxextra{main.MainWindow method}}

\begin{fulllineitems}
\phantomsection\label{\detokenize{main:main.MainWindow.Helpg2Button}}
\pysigstartsignatures
\pysiglinewithargsret{\sphinxbfcode{\sphinxupquote{Helpg2Button}}}{}{}
\pysigstopsignatures
\end{fulllineitems}

\index{about\_settings() (main.MainWindow method)@\spxentry{about\_settings()}\spxextra{main.MainWindow method}}

\begin{fulllineitems}
\phantomsection\label{\detokenize{main:main.MainWindow.about_settings}}
\pysigstartsignatures
\pysiglinewithargsret{\sphinxbfcode{\sphinxupquote{about\_settings}}}{}{}
\pysigstopsignatures
\sphinxAtStartPar
Opens the About window displaying information about the company, version, and repository.

\sphinxAtStartPar
This function opens a window showing details about the company (Tausand), the software version, 
and the location of the repository. It does not perform any checks or validations and directly 
displays the About window.

\sphinxAtStartPar
It does not take any parameters and does not return a value.
:returns: None

\end{fulllineitems}

\index{clicked\_tabs() (main.MainWindow method)@\spxentry{clicked\_tabs()}\spxextra{main.MainWindow method}}

\begin{fulllineitems}
\phantomsection\label{\detokenize{main:main.MainWindow.clicked_tabs}}
\pysigstartsignatures
\pysiglinewithargsret{\sphinxbfcode{\sphinxupquote{clicked\_tabs}}}{}{}
\pysigstopsignatures
\sphinxAtStartPar
Executes an action when a tab is clicked and creates the corresponding window.

\sphinxAtStartPar
This function checks which tab is currently active and constructs the associated window 
by invoking the appropriate function. If the tab corresponds to the Start/Stop Histogram, 
it stops the FLIM timer and constructs the Start/Stop Histogram window. If the tab corresponds 
to the Lifetime Measurements, it constructs the Lifetime window and sets up the FLIM logic if it 
has not been initialized yet.

\sphinxAtStartPar
It does not take any parameters and does not return a value.
:returns: None

\end{fulllineitems}

\index{closeEvent() (main.MainWindow method)@\spxentry{closeEvent()}\spxextra{main.MainWindow method}}

\begin{fulllineitems}
\phantomsection\label{\detokenize{main:main.MainWindow.closeEvent}}
\pysigstartsignatures
\pysiglinewithargsret{\sphinxbfcode{\sphinxupquote{closeEvent}}}{\sphinxparam{\DUrole{n,n}{event}}}{}
\pysigstopsignatures
\sphinxAtStartPar
Handles the close event of the main window and prompts the user for confirmation.

\sphinxAtStartPar
This function is triggered when the user attempts to close the main window. It displays 
a dialog asking if the user is sure about closing the application (Tempico software). 
If the user confirms by selecting “Yes”, the event is accepted and the application closes. 
If the user selects “No”, the close event is ignored, and the application remains open.
\begin{quote}\begin{description}
\sphinxlineitem{Parameters}
\sphinxAtStartPar
\sphinxstyleliteralstrong{\sphinxupquote{event}} (\sphinxstyleliteralemphasis{\sphinxupquote{QCloseEvent}}) \textendash{} The close event triggered when attempting to close the window.

\sphinxlineitem{Returns}
\sphinxAtStartPar
None

\end{description}\end{quote}

\end{fulllineitems}

\index{construct\_g2() (main.MainWindow method)@\spxentry{construct\_g2()}\spxextra{main.MainWindow method}}

\begin{fulllineitems}
\phantomsection\label{\detokenize{main:main.MainWindow.construct_g2}}
\pysigstartsignatures
\pysiglinewithargsret{\sphinxbfcode{\sphinxupquote{construct\_g2}}}{\sphinxparam{\DUrole{n,n}{parent}}}{}
\pysigstopsignatures
\sphinxAtStartPar
Constructs the g2 Measurements window.

\sphinxAtStartPar
This function takes a \sphinxtitleref{QTabWidget} parent, and if the sentinel is not set, it creates 
an instance of the \sphinxtitleref{Uig2} class and sets up the UI using the given parent.

\sphinxAtStartPar
It does not return a value.
\begin{quote}\begin{description}
\sphinxlineitem{Parameters}
\sphinxAtStartPar
\sphinxstyleliteralstrong{\sphinxupquote{parent}} (\sphinxstyleliteralemphasis{\sphinxupquote{QWidget}}) \textendash{} The parent widget (typically a \sphinxtitleref{QTabWidget}) for the lifetime measurements window.

\sphinxlineitem{Returns}
\sphinxAtStartPar
None

\end{description}\end{quote}

\end{fulllineitems}

\index{construct\_lifetime() (main.MainWindow method)@\spxentry{construct\_lifetime()}\spxextra{main.MainWindow method}}

\begin{fulllineitems}
\phantomsection\label{\detokenize{main:main.MainWindow.construct_lifetime}}
\pysigstartsignatures
\pysiglinewithargsret{\sphinxbfcode{\sphinxupquote{construct\_lifetime}}}{\sphinxparam{\DUrole{n,n}{parent}}}{}
\pysigstopsignatures
\sphinxAtStartPar
Constructs the Lifetime Measurements window.

\sphinxAtStartPar
This function takes a \sphinxtitleref{QTabWidget} parent, and if the sentinel is not set, it creates 
an instance of the \sphinxtitleref{UiFLIM} class and sets up the UI using the given parent.

\sphinxAtStartPar
It does not return a value.
\begin{quote}\begin{description}
\sphinxlineitem{Parameters}
\sphinxAtStartPar
\sphinxstyleliteralstrong{\sphinxupquote{parent}} (\sphinxstyleliteralemphasis{\sphinxupquote{QWidget}}) \textendash{} The parent widget (typically a \sphinxtitleref{QTabWidget}) for the lifetime measurements window.

\sphinxlineitem{Returns}
\sphinxAtStartPar
None

\end{description}\end{quote}

\end{fulllineitems}

\index{construct\_start\_stop\_histogram() (main.MainWindow method)@\spxentry{construct\_start\_stop\_histogram()}\spxextra{main.MainWindow method}}

\begin{fulllineitems}
\phantomsection\label{\detokenize{main:main.MainWindow.construct_start_stop_histogram}}
\pysigstartsignatures
\pysiglinewithargsret{\sphinxbfcode{\sphinxupquote{construct\_start\_stop\_histogram}}}{\sphinxparam{\DUrole{n,n}{parent}}}{}
\pysigstopsignatures
\sphinxAtStartPar
Constructs the Start/Stop Histogram window.

\sphinxAtStartPar
This function takes a \sphinxtitleref{QTabWidget} parent, and if the sentinel is not set, it creates
an instance of the \sphinxtitleref{Ui\_HistogramaStartStop} class and sets up the UI using the given parent.

\sphinxAtStartPar
It does not return a value.
\begin{quote}\begin{description}
\sphinxlineitem{Parameters}
\sphinxAtStartPar
\sphinxstyleliteralstrong{\sphinxupquote{parent}} (\sphinxstyleliteralemphasis{\sphinxupquote{QWidget}}) \textendash{} The parent widget (typically a \sphinxtitleref{QTabWidget}) for the histogram window.

\sphinxlineitem{Returns}
\sphinxAtStartPar
None

\end{description}\end{quote}

\end{fulllineitems}

\index{disconnect\_button\_click() (main.MainWindow method)@\spxentry{disconnect\_button\_click()}\spxextra{main.MainWindow method}}

\begin{fulllineitems}
\phantomsection\label{\detokenize{main:main.MainWindow.disconnect_button_click}}
\pysigstartsignatures
\pysiglinewithargsret{\sphinxbfcode{\sphinxupquote{disconnect\_button\_click}}}{}{}
\pysigstopsignatures
\sphinxAtStartPar
Handles the disconnect button click event.

\sphinxAtStartPar
This function hides the graphical display, disables the disconnect button, and re\sphinxhyphen{}enables the connect button. 
It also closes the connected device and resets its reference to \sphinxtitleref{None}. 
If additional graphics like \sphinxtitleref{g2Graphic} or \sphinxtitleref{FLIMGraphic} are active, it will disconnect them as well.

\sphinxAtStartPar
It does not take any parameters and does not return a value.
:returns: None

\end{fulllineitems}

\index{general\_settings\_clicked() (main.MainWindow method)@\spxentry{general\_settings\_clicked()}\spxextra{main.MainWindow method}}

\begin{fulllineitems}
\phantomsection\label{\detokenize{main:main.MainWindow.general_settings_clicked}}
\pysigstartsignatures
\pysiglinewithargsret{\sphinxbfcode{\sphinxupquote{general\_settings\_clicked}}}{}{}
\pysigstopsignatures
\sphinxAtStartPar
Opens the general settings window for device\sphinxhyphen{}wide configurations.

\sphinxAtStartPar
This function checks if a device is connected. If no measurement is currently running, 
it opens the general settings window where configurations that affect the entire device 
(regardless of the channel) can be adjusted. If a measurement is running, a message box 
is displayed to inform the user that changes cannot be made while the measurement is in 
progress. If no device is connected, a message box is shown indicating that no device was found.

\sphinxAtStartPar
It does not take any parameters and does not return a value.
:returns: None

\end{fulllineitems}

\index{open\_dialog() (main.MainWindow method)@\spxentry{open\_dialog()}\spxextra{main.MainWindow method}}

\begin{fulllineitems}
\phantomsection\label{\detokenize{main:main.MainWindow.open_dialog}}
\pysigstartsignatures
\pysiglinewithargsret{\sphinxbfcode{\sphinxupquote{open\_dialog}}}{}{}
\pysigstopsignatures
\sphinxAtStartPar
Opens a dialog window to detect and connect a measurement device.

\sphinxAtStartPar
This function creates and displays a dialog window that lists available measurement devices and their corresponding ports. 
Users can select a device and either connect or cancel the operation.

\sphinxAtStartPar
If the ‘Connect’ button is clicked:
\sphinxhyphen{} The selected device is connected, and all relevant software options are enabled.
\sphinxhyphen{} The function tries to open the device and, upon success, interacts with graphical components (if any are present) to reflect the connected status.

\sphinxAtStartPar
If the ‘Cancel’ button is clicked:
\sphinxhyphen{} No options are activated.

\sphinxAtStartPar
In case of a connection error, a message box is displayed informing the user of the failure.
\begin{quote}\begin{description}
\sphinxlineitem{Returns}
\sphinxAtStartPar
None

\end{description}\end{quote}

\end{fulllineitems}

\index{parameters\_action() (main.MainWindow method)@\spxentry{parameters\_action()}\spxextra{main.MainWindow method}}

\begin{fulllineitems}
\phantomsection\label{\detokenize{main:main.MainWindow.parameters_action}}
\pysigstartsignatures
\pysiglinewithargsret{\sphinxbfcode{\sphinxupquote{parameters\_action}}}{}{}
\pysigstopsignatures
\end{fulllineitems}

\index{resizeEvent() (main.MainWindow method)@\spxentry{resizeEvent()}\spxextra{main.MainWindow method}}

\begin{fulllineitems}
\phantomsection\label{\detokenize{main:main.MainWindow.resizeEvent}}
\pysigstartsignatures
\pysiglinewithargsret{\sphinxbfcode{\sphinxupquote{resizeEvent}}}{\sphinxparam{\DUrole{n,n}{event}}}{}
\pysigstopsignatures
\sphinxAtStartPar
Handles the window resize event and adjusts the column widths of the parameters table.

\sphinxAtStartPar
This function is triggered when the user resizes the main window. It resizes the columns 
of the parameters table in the FLIM tab based on the current window width. The column 
widths are scaled proportionally to ensure the table adapts to the new window size.
\begin{quote}\begin{description}
\sphinxlineitem{Parameters}
\sphinxAtStartPar
\sphinxstyleliteralstrong{\sphinxupquote{event}} (\sphinxstyleliteralemphasis{\sphinxupquote{QResizeEvent}}) \textendash{} The resize event triggered when the window is resized.

\sphinxlineitem{Returns}
\sphinxAtStartPar
None

\end{description}\end{quote}

\end{fulllineitems}

\index{settings\_clicked() (main.MainWindow method)@\spxentry{settings\_clicked()}\spxextra{main.MainWindow method}}

\begin{fulllineitems}
\phantomsection\label{\detokenize{main:main.MainWindow.settings_clicked}}
\pysigstartsignatures
\pysiglinewithargsret{\sphinxbfcode{\sphinxupquote{settings\_clicked}}}{}{}
\pysigstopsignatures
\sphinxAtStartPar
Opens the settings window when the settings option is clicked.

\sphinxAtStartPar
This function checks if a device is connected. If no measurement is currently running, 
it displays the settings window for channel configuration. If a measurement is running, 
a message box is shown informing the user that changes cannot be made while a measurement 
is in progress. If no device is connected, a message box alerts the user that no device 
was found.

\sphinxAtStartPar
It does not take any parameters and does not return a value.
:returns: None

\end{fulllineitems}

\index{staticMetaObject (main.MainWindow attribute)@\spxentry{staticMetaObject}\spxextra{main.MainWindow attribute}}

\begin{fulllineitems}
\phantomsection\label{\detokenize{main:main.MainWindow.staticMetaObject}}
\pysigstartsignatures
\pysigline{\sphinxbfcode{\sphinxupquote{staticMetaObject}}\sphinxbfcode{\sphinxupquote{\DUrole{w,w}{  }\DUrole{p,p}{=}\DUrole{w,w}{  }\textless{}PySide2.QtCore.QMetaObject object\textgreater{}}}}
\pysigstopsignatures
\end{fulllineitems}


\end{fulllineitems}

\index{SplashScreen (class in main)@\spxentry{SplashScreen}\spxextra{class in main}}

\begin{fulllineitems}
\phantomsection\label{\detokenize{main:main.SplashScreen}}
\pysigstartsignatures
\pysigline{\sphinxbfcode{\sphinxupquote{class\DUrole{w,w}{  }}}\sphinxcode{\sphinxupquote{main.}}\sphinxbfcode{\sphinxupquote{SplashScreen}}}
\pysigstopsignatures
\sphinxAtStartPar
Bases: \sphinxcode{\sphinxupquote{QMainWindow}}

\sphinxAtStartPar
Splash screen class for the application.

\sphinxAtStartPar
This class creates a splash screen that displays an image for a brief period before fading out and transitioning to the main application window.

\sphinxAtStartPar
The key responsibilities of this class include:
\sphinxhyphen{} Displaying a splash screen with a specified image.
\sphinxhyphen{} Timing the display duration and handling the transition to the main application window.
\begin{quote}\begin{description}
\sphinxlineitem{Parameters}
\sphinxAtStartPar
\sphinxstyleliteralstrong{\sphinxupquote{parent}} (\sphinxstyleliteralemphasis{\sphinxupquote{QWidget}}\sphinxstyleliteralemphasis{\sphinxupquote{, }}\sphinxstyleliteralemphasis{\sphinxupquote{optional}}) \textendash{} The parent widget (optional).

\end{description}\end{quote}
\index{show\_main\_window() (main.SplashScreen method)@\spxentry{show\_main\_window()}\spxextra{main.SplashScreen method}}

\begin{fulllineitems}
\phantomsection\label{\detokenize{main:main.SplashScreen.show_main_window}}
\pysigstartsignatures
\pysiglinewithargsret{\sphinxbfcode{\sphinxupquote{show\_main\_window}}}{}{}
\pysigstopsignatures
\sphinxAtStartPar
Displays the main window of the application.

\sphinxAtStartPar
This function creates an instance of the main window (\sphinxtitleref{MainWindow}) and displays it.
After showing the main window, it closes the current window.

\sphinxAtStartPar
It does not take any parameters and does not return a value.
:returns: None

\end{fulllineitems}

\index{staticMetaObject (main.SplashScreen attribute)@\spxentry{staticMetaObject}\spxextra{main.SplashScreen attribute}}

\begin{fulllineitems}
\phantomsection\label{\detokenize{main:main.SplashScreen.staticMetaObject}}
\pysigstartsignatures
\pysigline{\sphinxbfcode{\sphinxupquote{staticMetaObject}}\sphinxbfcode{\sphinxupquote{\DUrole{w,w}{  }\DUrole{p,p}{=}\DUrole{w,w}{  }\textless{}PySide2.QtCore.QMetaObject object\textgreater{}}}}
\pysigstopsignatures
\end{fulllineitems}


\end{fulllineitems}

\index{execProgram() (in module main)@\spxentry{execProgram()}\spxextra{in module main}}

\begin{fulllineitems}
\phantomsection\label{\detokenize{main:main.execProgram}}
\pysigstartsignatures
\pysiglinewithargsret{\sphinxcode{\sphinxupquote{main.}}\sphinxbfcode{\sphinxupquote{execProgram}}}{}{}
\pysigstopsignatures
\end{fulllineitems}



\chapter{Indices and tables}
\label{\detokenize{index:indices-and-tables}}\begin{itemize}
\item {} 
\sphinxAtStartPar
\DUrole{xref,std,std-ref}{genindex}

\item {} 
\sphinxAtStartPar
\DUrole{xref,std,std-ref}{modindex}

\item {} 
\sphinxAtStartPar
\DUrole{xref,std,std-ref}{search}

\end{itemize}


\renewcommand{\indexname}{Python Module Index}
\begin{sphinxtheindex}
\let\bigletter\sphinxstyleindexlettergroup
\bigletter{a}
\item\relax\sphinxstyleindexentry{aboutDialog}\sphinxstyleindexpageref{aboutDialog:\detokenize{module-aboutDialog}}
\indexspace
\bigletter{c}
\item\relax\sphinxstyleindexentry{createsavefile}\sphinxstyleindexpageref{createsavefile:\detokenize{module-createsavefile}}
\indexspace
\bigletter{f}
\item\relax\sphinxstyleindexentry{findDevices}\sphinxstyleindexpageref{findDevices:\detokenize{module-findDevices}}
\item\relax\sphinxstyleindexentry{FLIMGraphics}\sphinxstyleindexpageref{FLIMGraphics:\detokenize{module-FLIMGraphics}}
\indexspace
\bigletter{g}
\item\relax\sphinxstyleindexentry{generalsettings}\sphinxstyleindexpageref{generalsettings:\detokenize{module-generalsettings}}
\indexspace
\bigletter{m}
\item\relax\sphinxstyleindexentry{main}\sphinxstyleindexpageref{main:\detokenize{module-main}}
\indexspace
\bigletter{s}
\item\relax\sphinxstyleindexentry{StartStopHist}\sphinxstyleindexpageref{StartStopHist:\detokenize{module-StartStopHist}}
\end{sphinxtheindex}

\renewcommand{\indexname}{Index}
\printindex
\end{document}